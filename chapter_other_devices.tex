\chapter{Additional Carbon Nanotube Devices}
\label{sec:other_devices}
\chaptermark{Additional Devices}

\section{NM-CNT-NM Quantum Dots}

Palladium leads showed very low room temperature resistances of about \SI{20}{\kilo\ohm}. It was thought that by using a metal that makes very good contacts, fabrication problems not dependent on contact material might be easier to pinpoint. A few of these samples were made and measured in our dunker at 4K. 

The room temperature gate voltage behavior can be seen in Figure \ref{fig:PdQD_rt}(a). This quantum dot was made on a small gap semiconducting nanotube. This sort of gate curve was typical of palladium contacted devices; suggesting we grow a majority of small band gap semiconducting nanotubes. This is likely due to the diameter of the catalyst particles used in nanotube growth.

\begin{figure}
    \centering
    \includegraphics[width=0.9\textwidth]{other_devices/PdQD_RT.pdf}
    \caption{(a) Room temperature gate sweep of a palladium contacted quantum dot at positive bias voltage. (b) Resistances versus temperature curve for the same device. Coulomb blockade effects begin to dominate the transport below \SI{10}{\kelvin}.}
    \label{fig:PdQD_rt}
\end{figure}

At low temperatures these devices showed weak Coulomb blockade behavior. Low temperature measurements of the device measured in Figure \ref{fig:PdQD_rt} are seen in Figure \ref{fig:PdQD_4K}. The charging and excitation energies are consistent with other devices measured in this work. Figure \ref{fig:PdQD_4K}(b) shows conductance peaks in the gate sweep have irregular spacing and no clear four fold symmetry. This suggests the quantum dot has impurities along its length causing more than one quantum dot to act in series along the nanotube \cite{Bockrath2001}. Again, this is consistent with other devices measured and suggests defects in the nanotube growth or contamination on the substrate near the nanotube.

\begin{figure}
    \centering
    \includegraphics[width=0.9\textwidth]{other_devices/PdQD_4K.pdf}
    \caption{(a) Conductance as a function of $V_{gate}$ in a Pd contacted nanotube quantum dot at 4K. (b) Conductance as a function of $V_{bias}$ and $V_{gate}$ taken in the region marked by dashed lines in (a).}
    \label{fig:PdQD_4K}
\end{figure}

\section{Locally Gated Quantum Dots}

Due to the nature of the Schottky barriers between metals and carbon nanotubes, it can be difficult to tune the levels on a nanotube quantum dot without affecting the barrier height and charging energy of the dot \cite{Schottky1938, Svensson2011}. One solution to better control of the quantum dot is to add narrow local gates to the devices. This is most easily done by adding side gates or top gates. Figure \ref{fig:local_gating} shows an example of each of these device geometries.

\begin{figure}
    \centering
    \includegraphics[width=0.9\textwidth]{other_devices/gated_device_images.pdf}
    \caption{Two local gating techniques. Left: A top gated devices with a low temperature ALD grown \ce{Al2O3} layer separating the top gate from the underlaying nanotube. Right: a side gated devices with narrow Cr/Au leads positioned between the Co leads \SI{200}{\nano\meter} from the nanotube.}
    \label{fig:local_gating}
\end{figure}

Contacting nanotubes with ferromagnetic and superconducting metals proved so difficult, adding an additional fabrication set to position local gates was deemed too much of a risk in most cases. A few devices were measured toward the end of this thesis work. Typical room temperature and 4K data are seen in Figures \ref{fig:local_gate_rt} and \ref{fig:local_gate_4K}.

\begin{figure}
    \centering
    \includegraphics[width=0.9\textwidth]{other_devices/local_gate_rt.pdf}
    \caption{Left: Current as a function of $V_{BG}$. Right: Current as a function of $V_{TG}$. Even in this simple example it is clear that the local gates are much more effective at tuning the Fermi level on the carbon nanotube.}
    \label{fig:local_gate_rt}
\end{figure}

\begin{figure}
    \centering
    \includegraphics[width=0.75\textwidth]{other_devices/local_gate_4K.pdf}
    \caption{Top: Current as a function of $V_{TG}$ and $V_{BG}$. Bottom: A cut across the plot demonstrating how both gates can be used together to keep the tunnel barrier height fixed while varying the Fermi level on the quantum dot.}
    \label{fig:local_gate_4K}
\end{figure}

Figure \ref{fig:local_gate_4K}(top) shows how the resonant tunneling peaks behave as a function of both gates. The dashed white line across the plot is perpendicular to the path followed by the conductance peaks. By tuning the gates along this line, the tunnel barriers in the source and drain contacts can be held constant while the energy levels on the dot are varied relative to the Fermi level on the leads. A plot of the measured current along this detuning axis can be seen in the bottom of Figure \ref{fig:local_gate_4K}.

\section{CNT Tunnel Probe into Aluminum Nanowires}

Carbon nanotubes, because of their small diameter and high contact resistances with metals, make good candidates for tunnel probes into other mesoscopic systems. In this project, nanotubes were used in an attempt to probe the density of states of a quasi 1D aluminum wire. A schematic of the device being discussed can be seen in Figure \ref{fig:cnt_al_device}

\begin{figure}
    \centering
    \includegraphics[width=0.5\textwidth]{other_devices/cnt_al_device.pdf}
    \caption{Schematic of a CNT tunnel probe under two narrow aluminum wires.}
    \label{fig:cnt_al_device}
\end{figure}

\subsection{Background}

Aluminum is a superconductor with a bulk transition temperature of 1.2K. When placed in a perpendicular magnetic field, thin films of aluminum allow magnetic vortices to pass through in the superconducting state. These vortices have a diameter on the order of the superconducting coherence length \cite{Tinkham1996}. When the width of the aluminum film is made to be on the order of the coherence length, these vortices are forced to pass through the superconductor in a single line. This geometric confinement makes it possible to tune the number of vortices trapped in the nanowire by varying the applied magnetic field. This effect is called the Weber blockade \cite{Pekker2011}.

\begin{figure}
    \centering
    \includegraphics[width=0.9\textwidth]{other_devices/figure4_4.png}
    \caption{Figure taken from Ref \cite{Morgan-Wall2015}. (a) Potential seen by vortices along width of the aluminum nanowire as a function of applied perpendicular magnetic field. (b) Potential seen by vortices along width of the aluminum nanowire as a function of applied current. (c) Critical current as a function of applied perpendicular magnetic field. The Weber blockade region can be seen clearly in Region III.}
    \label{fig:aluminum_potential}
\end{figure}

Figure \ref{fig:aluminum_potential} is taken from our lab's publication on Weber blockade in aluminum nanowires \cite{Morgan-Wall2015}. In Figure \ref{fig:aluminum_potential}(a) the potential seen by a vortex is plotted as a function of the distance across the wire. A similar plot is shown in Figure \ref{fig:aluminum_potential}(b) Showing the potential across the wire as a function of the applied current. By tuning these parameters into Region III, as labelled in Figure \ref{fig:aluminum_potential}(c), vortices will hop from the vacuum onto the nanowire one at a time. The varying number of vortices can be identified by changes in the critical current of the nanowire, as seen in Region III of Figure \ref{fig:aluminum_potential}(c). This is analogous to the Coulomb blockade in quantum dots. In fact, if the critical current is plotted as a function of applied field, we find the data looks just like a Coulomb blockade plot, with the blockaded regions being replaced by regions of superconductivity containing a fixed number of vortices. This measurement can be seen in Figure \ref{fig:weber_measurement}.

\begin{figure}
    \centering
    \includegraphics[width=0.9\textwidth]{other_devices/figure4_3.png}
    \caption{Figure taken from Ref \cite{Morgan-Wall2015}. (a) IV curves as a function of applied magnetic field. The Weber blockade region can be clearly seen around 100mT. (b) Close up of the Weber blockade region. (c) Weber blockade effect with superconducting regions containing a fixed number of vortices seen in blue.}
    \label{fig:weber_measurement}
\end{figure}

Because the vortices repel one another, their spacing is dependent on the number of vortices contained along the length of the nanowire. If a nanotube is placed at a fixed position under the aluminum wire, the tunnel current will depend on whether or not a vortex is positioned above the nanotube. Thus, by measuring the differential conductance through the nanotube tunnel probe, the position of vortices in the wire can be mapped as a function of applied magnetic field.

\subsection{Device and Measurement}

Several of these tunnel probe devices were created by placing aluminum nanowires across palladium carbon nanotube quantum dots. The fact that these were quantum dots is not really important, but does provide a means of judging the quality of the nanotube portion of the devices. Typical measurements of Pd contacted nanotube quantum dots under aluminum nanowires are seen in Figure \ref{fig:tunneling_device}.

\begin{figure}
    \centering
    \includegraphics[width=0.75\textwidth]{other_devices/tunneling_device.pdf}
    \caption{(a) SEM image of a typical aluminum wire tunnel probe device. The Pd leads connected to the nanotube form a quantum dot under the device. (b) Conductance through the nanotube quantum dot with aluminum nanowire on top. (c) Conductance as a function of back gate voltage through the same devices.}
    \label{fig:tunneling_device}
\end{figure}

No useful measurements of the tunnel probe were made, mainly because of poor contact to the nanotube. The first few devices suffered from the Al-CNT resistances being larger than the input resistance of my current to voltage amplifier. This was improved in subsequent devices by adding thin titanium layers beneath the aluminum nanowire. However, later devices showed poor contact resistances between normal leads and nanotubes. With a little more time, I see no reason this project would not yield the predicted results.

\section{Majorana Ferimons in CNTs}
\label{sec:majorana}

For a time, around 2010, finding Majorana fermions in condensed matter systems was a source of much excitement. Naturally, this lead to attempts to create a Majorana mode in a carbon nanotube device. Below is a brief description of the project and a summary of the results.

\subsection{Background}

Majorana fermions, first predicted by Ettore Majorana in 1937\cite{Majorana1937}, can be thought of as charge-neutral particles that have the unique property of being their own anti-particle. Majorana first proposed these states as an explanation for neutrinos. Recently, much excitement has been generated by a number of papers proposing similar states may be created in solid state systems\cite{Lutchyn2010, Alicea2010}. I am particularly interested in measuring these states as they are predicted to exist in semiconductor-superconductor heterostructures.

Some basic insight can be gained from examining a toy model proposed by Kitaev\cite{Kitaev2001}.  By writing the Hamiltonian for a one-dimensional wire with N sites, a single spin component, and a superconducting gap, the model allows for the existence of Majorana modes within certain parts of the parameter space.  

\begin{equation}
	H = \sum_{j=1}^{N}\left(-w\left(a_j^\dagger a_{j+1}+a_{j+1}^\dagger a_j\right)-\mu\left( a_j^\dagger a_{j} -\frac{1}{2}\right) + \right| \Delta \left| e^{i\theta} a_ja_{j+1} + \left|\Delta\right|e^{-i\theta} a_{j+1}^\dagger a_j^\dagger\right)
\end{equation}

The Hamiltonian can be rewritten by splitting each fermion site into two half-fermion Majorana sites with the operators $c_{2j-1}=e^{i\frac{\theta}{2}}a_j+e^{-i\frac{\theta}{2}}a_j^\dagger$ and $c_{2j}=-i\left(e^{i\frac{\theta}{2}}a_j-ie^{-i\frac{\theta}{2}}a_j^\dagger\right)$.  Considering the new Hamiltonian in two simple limits reveals two ground states. In the trival case, $\left|\Delta\right| = w = 0,\quad \mu < 0$, one finds a ground state with no unpaired Majorana modes.

\begin{equation}
	H = -\mu\sum_{j=1}^{N}\left(a_j^\dagger a_j - \frac{1}{2}\right) = \frac{i}{2}\mu \sum_{j=1}^{N} c_{2j-1}c_{2j}
\end{equation}

In the case of non-zero hopping and a non-zero superconducting gap, $\left|\Delta\right| = w > 0,\quad \mu = 0$, one finds a ground state with coupling between Majorana operators on different sites.  This leaves two unpaired Majorana modes at the ends of the wire (sites 1 and N).  

\begin{equation}
	H = 2w\sum_{j=1}^{N-1}\left(\tilde{a}_j^\dagger \tilde{a}_j - \frac{1}{2}\right) = iw\sum_{j=1}^{N} c_{2j}c_{2j+1}
\end{equation}

The Majorana modes can also be shown to exist for ranges of $\mu$, $\Delta$, and $w$ outside of the trivial cases discussed above.  

\subsection{Device and Measurement}

The Kitaev model has three requirements.  The model is specific to a one dimensional conductor, the conductor must have a superconducting gap, and the wire must support helical conduction modes (in which opposite spins are conducted in opposite directions).  It may be possible to satisfy all of these conditions in a carbon nanotube device.

Kitaev's first condition is satisfied by the choice of carbon nanotubes as the conducting medium.  Realizing the second condition in a nanotube device simply requires contacting the nanotube with a superconduting material, such as aluminum, which is most compatible with our fabrication techniques.  Cooper pairing then becomes possible within the nanotube through proximity induced superconductivity \cite{Kasumov1999, JarilloHerrero2006}. A summary of these conditions is seen in Figure \ref{fig:band_splitting}. 

It has recently been shown that carbon nanotubes may support helical conduction bands\cite{Klinovaja2011a, Klinovaja2011}.  When lattice curvature effects are calculated and some combination of external electric and magnetic fields applied the level splitting the carbon nanotube can be fixed such that helical bands exist for certain values of the chemical potential. Additionally, many measurements have been made confirming the spin-orbit coupling in carbon nanotubes \cite{Kuemmeth2008, Jespersen2011, Steele2013 ,Lai2014}.

\begin{figure}
    \centering
    \includegraphics[width=0.9\textwidth]{other_devices/majorana_band_splitting.pdf}
    \caption{(a) Approximate band structure of a semiconductor. (b) Band splitting along the k-axis due to spin-orbit coupling. (c) Additional Zeeman splitting along the E-axis leads to individual, helical conduction bands.}
    \label{fig:band_splitting}
\end{figure}

To detect the Majorana states, it should be possible to perform a simple tunneling experiment. The Majorana state should be created and destroyed as a function of the applied fields (change level splitting) and gate (back) voltage applied to the semiconducting wire (changes chemical potential). By adding an additional tunneling contact on top of the aluminum/CNT contact, the current through the tunneling contact and Al lead can be measured to determine the available conduction bands through the Al/CNT junction. With the creation of a Majorana state, a peak should be measured in the tunneling current at zero energy. Similar devices have been measured on \ce{InSb} nanowires \cite{Mourik2012}. 

A typical device is seen in Figure \ref{fig:majorana_data}. Several of these devices were measured at 250mK in an Oxford He3 cryostat. The devices did show superconducting quantum dot behavior, but none of the tunne probes behaved as desired. The probes were either shorted to the leads or showed a resistance larger than the current amplifier input. An example of the data taken on one of these devices is seen in Figure \ref{fig:majorana_data}

\begin{figure}
    \centering
    \includegraphics[width=0.9\textwidth]{other_devices/majorana_data.pdf}
    \caption{Left: A typical tunnel probe device with aluminum leads and a Cr/Au tunnel probe. Right: Data taken from a similar device with Niobium leads at 250mK.}
    \label{fig:majorana_data}
\end{figure}

These devices were measured very early on in my thesis work. The low resolution data, and very long data acquisition times, are what motivated a lot of future work to improve the measurement technique. Unfortunately, no more data was taken for this project.