%% FRONTMATTER

\begin{frontmatter}

\maketitle

\begin{abstract}

Carbon nanotube quantum dots are an attractive platform in which to measure quantum transport phenomena. Low energy transport properties of nanotubes are easily understood and the devices are simple to fabricate with a wide range of metal contact materials. Quantum transport in the dot is easily tuned by varying the length of the device and choice of material. By fabricating carbon nanotube quantum dots with ferromagnetic and superconducting contacts, it becomes possible to measure a wide variety of spin transport phenomena at low temperatures. In this thesis, I have studied the fabrication and low-temperature transport properties of carbon nanotube quantum dots with normal, ferromagnetic, and superconducting contacts. A wide range of fabrication techniques were tested and optimized along with improvements to image processing and contact fabrication.  F-CNT-F devices show a range of spin dependent physics, including tunneling magnetoresistance and suppression of conductance peaks due spin selection rules. These results offer a probe into the collective spin states in a CNTQD. F-CNT-S devices show evidence of proximity induced superconductivity and magnetic field dependent switching of the conductance. The measurements are the first attempt at analyzing conductance through a F-CNT-S quantum dot. The results presented in this thesis represent a step in improving device fabrication through statistical analysis and improved methods, as well as a look at spin dependent transport through a variety of carbon nanotube quantum dot structures.

\vspace{1cm}

\noindent Primary Reader: Nina Markovic\\
Secondary Reader: N. Peter Armitage

\end{abstract}

\begin{acknowledgment}

Over the past seven years, whenever I have been asked why I chose to do my PhD work at Johns Hopkins I have had the same answer; the people here were by far the nicest and most helpful scientists I spoke with in making my decision. That has remained true throughout my time at Johns Hopkins. With that, the biggest thanks goes to my advisor, Nina Markovi\'{c}. She has consistently brought in big ideas for each of the graduate students in her lab and given us the freedom to run wild with them. I could not have had an advisor with a personality I related to more than Nina. She was always there to put me back in my place the moment I complained rather than getting back to work in the lab. That alone has gone a long way to making me a better scientist. Along with Nina I owe thanks to Dan Reich, Peter Armitage, and Oleg Tchernyshyov for making sure I was alive, paid, and stocked with equipment in my last year.

During the time I spent in the Markovi\'{c} lab I learned a great deal from my labmates. Soo Hyung Lee was here at the beginning to help me navigate the unfamiliar lab. Janice Guikema, throughout my time here, was always willing to help me think about a problem or correct a sloppy technique. Similarly, our post doc Atikur Rahman provided huge amounts of measurement advice. I would not be able to cool down a single sample without what I learned from Atikur. Tyler Morgan-Wall has worked with me since nearly my first day in the lab. Rewiring cryostats, troubleshooting fabrication, and cooling a sample down until 2am would not have been half as much fun without him. JT Mlack, despite abandoning us for Copenhagen for a year, has been a great labmate to spend two years sitting uncomfortably close to in the basement. He was always on hand to listen to a rant about fabrication problems, or, just as likely, whatever science fiction we were into at the time. Finally, I have to acknowledge the hard work of all of the undergraduate and high school students who have worked with me over the years; Steph Blease, Joe Schwartz, Ben Hartman, Paul Bewak, Alec Jordan, and Streit Cunningham. They all deserve a lot of credit for carrying the team through long hours in the cleanroom, in front of the AFM, and watching over the tube furnace. This thesis would not have been possible without their dedication and insights.

In addition to my labmates, there are too many other physicists and Baltimoreans to thank. I owe so much to Dan Allan, Nuala McCullagh, and Dan Richmond for being here throughout my whole carrer to celebrate success, work through failure, and have a beer pretty much any time anything happened. To all of you who spent time with me at the Meat Castle, Countdown, and New Years cabins, it was phenomenal and I hope to see you all for years to come. Finally, thanks to every other physicist I've spent time with in Bloomberg. You've all done a great job at maintaining the spirit that brought me here in the first place.

Thanks to my family, my parents Ann and George, and my sister Erika, who were unfailingly supportive of my work on this PhD. Finally, thanks to Stephanie, who, to put it briefly, is the reason I made it to my defense with my sanity (along with Herman, of course).

\end{acknowledgment}

%\begin{dedication}
% 
%This thesis is dedicated to \ldots
%
%\end{dedication}

% generate table of contents
\tableofcontents

% generate list of tables
\listoftables

% generate list of figures
\listoffigures

\end{frontmatter}
