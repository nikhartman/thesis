%% FRONTMATTER

\begin{frontmatter}

\maketitle

\begin{abstract}

Carbon nanotube quantum dots are an attractive platform in which to measure quantum transport phenomena. Low energy transport properties of nanotubes are easily understood and the devices are simple to fabricate with a wide range of metal contact materials. Quantum transport in the dot is easily tuned by varying the length of the device and choice of material. By fabricating carbon nanotube quantum dots with ferromagnetic and superconducting contacts, it becomes possible to measure a wide variety of spin transport phenomena a low temperatures. In this thesis, I have studied the fabrication and low-temperature transport properties of carbon nanotube quantum dots with normal, ferromagnetic, and superconducting contacts. A wide range of fabrication techniques were tested and optimized along with improvements to image processing and contact fabrication.  F-CNT-F devices show a range of spin dependent physics, including tunneling magnetoresistance and suppression of conductance peaks due spin selection rules. These results offer a probe into the collective spin states in a CNTQD. F-CNT-S devices show evidence of proximity induced superconductivity and magnetic field dependent switching of the conductance. The measurements are the first attempt at analyzing conductance through a F-CNT-S quantum dot. The results presented in this thesis represent a step in improving device fabrication through statistical analysis and improved methods, as well as a look at spin dependent transport through a variety of carbon nanotube quantum dot structures.

\vspace{1cm}

\noindent Primary Reader: Nina Markovic\\
Secondary Reader: N. Peter Armitage

\end{abstract}

\begin{acknowledgment}

Thanks \ldots

%Over the past seven years, whenever I have been asked why I chose to do my PhD work at Johns Hopkins I have had the same answer; the people here were by far the nicest and most helpful scientists I spoke with in making my decision. That has remained true throughout my career here. With that, the biggest thanks goes to my advisor, Nina Markovic. She has consistently brought in big ideas for each of the graduate students in her lab, and given us the freedom to run wild with them. I could not have had an advisor with a personality I related to more than Nina. She was also always there to put me back in my place the moment I complained rather than getting back to work in the lab. That alone has gone a long way to making me a better scientist.
%
%During the time I spent in the Markovic lab I learned a great deal from my labmates. Soo Hyung Lee was here at the beginning to help me navigate the unfamiliar lab. Janice Guikema, throughout my time here, was always on hand to help me think about a problem or correct a sloppy technique. Similarly, our post doc Atikur Rahman provided huge amounts of measurement advice. I would not be able to cool down a single sample without what I learned from Atikur. Tyler Morgan-Wall has worked with me since nearly my first day in the lab. Rewiring cryostats, troubleshooting fabrication, and cooling a sample down until 2am would not have been half as much fun without him. JT Mlack came in a two years later, and was always on hand to discuss sci-fi, samples, and remind me I might take things a little too seriously. 
%
%In addition to my labmates, there are too many other physicists and Baltimoreans to thank. Thanks to Dan Allan, Dan Richmond, and Nuala McCullagh for suffering through the first year, and more, with me. Seamus Riley, Jess Brick, Laura McDonald, Sarah LaRocca, Andrew Whitbeck, and Stefan Byrd-Kreuger, our time at the Meat Castle, Countdown, and New Years cabins was phenomenal. Thanks to every other physicist in Bloomberg who has helped maintain the social, collaborative atmosphere that brought me here.
%
%Thanks to my family, my parents Ann and George, and my sister Erika, who were unfailingly supportive of my work on this PhD. Finally, thanks to Stephanie, who has put up with watching me struggle to finish the PhD for quite some time without ever losing her signature cheerful outlook.

\end{acknowledgment}

%\begin{dedication}
% 
%This thesis is dedicated to \ldots
%
%\end{dedication}

% generate table of contents
\tableofcontents

% generate list of tables
\listoftables

% generate list of figures
\listoffigures

\end{frontmatter}
