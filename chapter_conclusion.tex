\chapter{Conclusion}
\label{sec:conclusion}
\chaptermark{Conclusion}

Carbon nanotube devices have been widely discussed over the past two decades. Each year brings new publications discussing a wide variety of transport phenomena, applications, and fabrication techniques. This thesis adds to that pool by presenting a set of transport measurements on spin dependent tunneling process in F-CNT-F and F-CNT-S devices. Additionally, a large effort has been made to tabulate the success rates for various fabrication techniques tested. Despite the large volume of nanotube fabrication research currently available, there has been very little discussion of reproducibility and device yield. 

\section{Fabrication of Nanotube Devices}

Chapter \ref{sec:growth} and Appendix \ref{chap:fabrication} detail the methods used, and attempted, to produce carbon nanotube devices in our lab. An effort has been made to detail each fabrication method, as well as comment on those that were found to not be reproducible. Improvements to the fabrication, particularly in the area of nanotube imaging are presented. The imaging methods developed are intended to decrease the processing time required for each device and, as a result, decrease the risk of environmental contamination during imaging.

Chapter \ref{chap:contacts} extends the work in Chapter \ref{sec:growth} and presents a detailed discussion of the device yield and contact resistances of devices produced as a function of materials and methods used. It is my hope that this analysis will help future researchers in producing devices with more consistent results and less wasted fabrication effort.

A few guidelines for device fabrication can be derived from the information in those two chapters (combined with information from the transport measurements in later chapters):

Drop cast nanotube samples have a few advantages over substrate growth methods. First, they are not prone to the types of gate leak problems that were frequently seen in substrate grown samples due to the high temperature growth process. Second, although it is more difficult to find a nanotube of sufficient length to make a device with drop casting, the devices made may show fewer growth defects. The sonication required to suspend nanotubes appears to preferentially break tubes at defects resulting in a shorter average length, but with fewer defects in the long tubes that survive the process. Placing substrate grown tubes is a much more reliable process, with almost 100\% of growth chips having nanotubes suitable for device fabrication. The process also does not require the handling of bulk carbon nanotube material which is difficult as well as hazardous. When combined with image processing techniques in Chapter \ref{sec:growth}, the device fabrication process is significantly faster. Because of the disorder measured in F-CNT-F devices, device lengths are best kept to less than 200nm if a single, defect-free quantum dot is desired in a substrate grown nanotube. It is difficult to compare the device yield due to a low number of drop cast samples having been made for this work.

Once nanotubes have located on the substrate, fabricating clean, low resistance contacts becomes the major challenge. My work has found that PMMA combined with low temperature IPA:H20 development and over exposure of nanotube contact patterns produces the cleanest resist masks for contact deposition. Low-energy magnetron sputtering has proved to be the most reliable method for producing low-resistance nanotube contacts with normal metal, ferromagnetic, and superconducting materials. Control over the sample temperature may have a large effect on these results. Magnetron sputtering prevents samples from being exposed to free electron bombardment and keeps samples cool during deposition by avoiding high temperature melting of source materials. Future work may investigate this observation by testing thermal and electron beam evaporation with a temperature controlled sample stage to determine the optimum temperature range for deposition of nanotube contacts.

\section{Ferromagnetic and Superconducting Device Measurements}

Measurements on F-CNT-F devices show that it is possible to reproduce previously pubished results on gate-tunable tunnel magnetoresistance \cite{Sahoo2005, Aurich2010}. In the same devices, evidence is present to show that the spin states can be probed through transport spectroscopy and manipulated by application of an external magnetic field. Finally, a argument has been made for F-CNT-F devices as probes into the spin dependent structure of magnetic impurities. Future work should focus on the evolution of the spin states in these quantum dots in the presence of a magnetic field. By measuring the evolution of the ground state transitions between different quantum dot occupation numbers the nature of the change in spin state discussed in Section \ref{sec:spin_selection_field} could be further probed.

A brief analysis of two F-CNT-S devices is presented in Chapter \ref{sec:SCFM}. These devices show evidence for proximity effect superconductivity in the nanotube, despite the presence of the ferromagnetic contact. Additionally, a discussion of magnetic field dependent conductance features is presented. This effect lacks a clear physical origin and should be further investigated by measurements with gate-tunable F-CNT-S junctions.

\section{Future Work on Other Devices}

Chapter \ref{sec:other_devices} presents work done on additional carbon nanotube devices. The most promising avenue for future research is in the use of carbon nanotubes as tunnel probes into low-dimensional superconducting materials. This project was not completed due to time and equipment constraints, but shows great promise and sound theoretical footing. 

The Majorana device discussed in Section \ref{sec:majorana} is very interesting from a theoretical standpoint. From the view of an experimentalist it is very difficult. The device was not pursued further in this work due to difficulties in fabricating the junctions as well as the noise limits of our experimental setup. Properly measuring the zero bias conductance peak predicted in the theory will require very low electron temperatures and accurate conductance measurements to make the results believable. A more realistic proposal might involve two tunable nanotube devices in a SQUID geometry in which the Majorana mode can be probed by measuring changes to the period in the critical current oscillations \cite{Wang2015}. 