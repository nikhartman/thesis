\chapter{Metallic Contacts to Carbon Nanotubes}
\label{sec:contacts}
\chaptermark{Contacts to CNTs}

In the course of making the devices measured in later chapters of this thesis, many carbon nanotube quantum dots were tested in our room temperature probe station. It became clear that there were problems with producing devices with consistent two-probe resistances. The only two possible sources for the anomalously large resistances (>\SI{100}{\mega\ohm})are poor quality nanotubes and large contact resistances between thin films deposited for source and drain contacts. 

All devices measured for this chapter were created by growing nanotubes directly on silicon substrates using the methods discussed in Section \ref{sec:catalyst_island}. The quality of these nanotube was tested with several measurements. Atomic force microscope measurements of the tube diameter confirmed nanotubes were individual, single-walled nanotubes with typical diameters around \SI{2}{\nano\meter}. Scanning Electron Microscope measurements checked the nanotube lengths and density. Both AFM and SEM measurements were used to select only substrates that did now show excessive amorphous carbon on the substrate after CVD growth. Finally, electrical measurements of three-terminal devices with low contact resistances showed the expected gate dependence, as seen in in Figure \ref{fig:fet_measurements}.

The apparent high quality of the nanotubes left the metal/nanotube interfaces as the most obvious source of the high resistances measured in many two-probe devices. More importantly, devices did not show a broad distribution of contact resistances. Rather, when a silicon chip had working devices, it tended to have many, low resistance (<\SI{10}{\mega\ohm}) devices. This suggests that some step in the fabrication on a chip wide scale was responsible the large metal/nanotube contact resistances. Additional evidence for contamination and impurities, as seen in transport measurements, is discussed at the end of this chapter

To determine the source of the contamination, detailed notes were kept on the fabrication methods and resulting probe station measurements for over 300 silicon chips (one chip typically contains tens of possible quantum dot devices). The first sections of this chapter discuss the results of several fabrication methods. Several recommendations are made for producing more reliable carbon nanotube quantum dots.

\section{Statistics of Carbon Nanotube Contacts}

The clearest evidence for contamination was in the almost binary nature of the two probe nanotube resistances measured in our probe station. If a chip showed low resistances on one device, that typically held for multiple devices on the chip, even those built on different nanotubes. Based on this observation, the first statistic checked was the yield of devices with resistances less than \SI{1}{\giga\ohm}. Figure \ref{fig:quantum_dot_yield} shows that data, separated by the type of metals used to create the two terminal nanotube device measured. This data includes all metals, all lithography methods, and all substrate preparations.

\begin{figure}
    \centering
    \includegraphics[width=0.6\textwidth]{chapter4/quantum_dot_yield.png}
    \caption{Pie charts showing the percentage of quantum dots with measured resistances below \SI{1}{\giga\ohm}. Data includes all metals and deposition methods.}
    \label{fig:quantum_dot_yield}
\end{figure}

Figure \ref{fig:rt_resistances} shows the distribution of all devices with resistances less than \SI{1}{\giga\ohm} from the same data set as Figure \ref{fig:quantum_dot_yield}. The results in both Figure \ref{fig:quantum_dot_yield} and \ref{fig:rt_resistances} show similar results. Normal metals are most reliable, this is likely because the best choice of normal metal is a well researched problem \cite{Javey2003, Song2011, Kane2011}. As will be seen later, sputtered palladium was consistently the best contact material tested. Ferromagnetic materials were the most frequently tested in this work. When the devices did work, they show a distribution of contact resistances that skews lower than the superconducting metals. Superconductor-ferromagnet junctions suffer from higher resistance superconducting contacts and an additional fabrication step, both of which contribute to the lower yield and higher average resistances for those devices.

\begin{figure}
    \centering
    \includegraphics[width=0.6\textwidth]{chapter4/room_temp_cntqd_resistances.png}
    \caption{Histograms showing distribution of devices resistances. Only measured resistances below \SI{1}{\giga\ohm} are plotted. Data includes all metals and deposition methods.}
    \label{fig:rt_resistances}
\end{figure}

Table \ref{table:all_contact_results} summarizes all of the materials and deposition methods tested. Many of these materials and methods were also tested with a range of lithography parameters as discused in Section \ref{subsec:developer_choice}. 

\begin{table}
    \centering
    \caption{Summary of all materials and deposition methods tested for nanotube contacts.}
    \begin{tabular}{ r | c c c}
        Material & Method & Yield & Total Number Produced \\ \hline
        Cobalt & Thermal & 19.6\% & 392 \\ 
        Cobalt & Electron Beam & 33.9\% & 109 \\ 
        Cobalt & Sputtered & 24.7\% & 223 \\ 
        Permalloy & Thermal & 54.5\% & 11 \\
        Permalloy & Sputtered & 13.0\% & 23 \\
        Aluminum & Thermal & 19.2\% & 167 \\
        Niobium & Sputtered & 62.5\% & 8 \\
        Gold & Thermal & 48.2\% & 56 \\
        Palladium & Sputtered & 56.7\% & 67
    \end{tabular}
    \label{table:all_contact_results}  
\end{table}

\subsection{Ferromagnetic Contacts}

Ferromagnetic materials comprise the largest portion of the data set collected for this analysis. This is because they were the first materials tested, and the widest variety of materials was available. No sticking layers were used in any of the depositions discussed here for fear of ruining the spin transport properties of the metal/nanotube interface.

\begin{figure}
    \centering
    \includegraphics[width=0.9\textwidth]{chapter4/fm_cnt_contacts.png}
    \caption{Histograms showing distribution of ferromagetic metal contact resistances. Insets show the yield of devices with resistance below \SI{1}{\giga\ohm}}
    \label{fig:fm_contacts}
\end{figure}

Two claims are made based on the results summarized in Figure \ref{fig:fm_contacts}. First, sputtered cobalt consistently makes good contact to nanotubes. Second, electron beam evaporation produces lower device resistance devices, which is supported by previous work \cite{Churchill2012}. Only a small number of electron beam evaporated devices were tested due to limited access to the evaporator.

The permalloy plots are hiding an important detail. Most thermally evaporated permalloy devices did not have a measurable resistance until after they were annealed at \SI{300}{\celsius} for 3 hours in an Ar/\ce{H2} atmosphere. This lowered the resistances to the values recorded in this data set. However, annealing under these conditinos often introduced gate leaks in the devices, making the method useless for this work.

\subsection{Superconducting Contacts}

Superconducting contacts were first measured on carbon nanotubes in 2006. In that work Ti/Al was used as the superconducting layer. Aluminum is a cheap superconducting material with a relatively long coherence length, making it useful in proximity effect devices, such as nanotube quantum dots. The other commonly used superconductor in nanotube devices is niobium or a Ti/Nb bilayer. Niobium has the advantage of having a transition temperature above 4K, making it possible to measure in a simple liquid helium dunker. The Ti layers, in each case, are added because titanium is thought to make good contact with carbon nanotubes by wetting and forming titanium carbide compounds on the nanotube surface. Ti/Al and Ti/Nb bilayers were tested for this work, with the titanium layer being deposited in the same way as the superconductor without breaking vacuum. Despite the difficulty in thermally evaporating titanium, it was necessary here because chromium introduces some trace magnetic contamination to the superconductoring material that was deemed unacceptable.

\begin{figure}
    \centering
    \includegraphics[width=0.6\textwidth]{chapter4/sc_cnt_contacts.png}
    \caption{Histograms showing distribution of superconducting metal contact resistances. Insets show the yield of devices with resistance below \SI{1}{\giga\ohm}}
    \label{fig:sc_contacts}
\end{figure}

The results in Figure \ref{fig:sc_contacts} are difficult to interpret. The Ti/Al layers have much better statistics, but the Ti/Nb results seem to suggest a better device yield. It is suspected that this is due to poor thermal evaporation of the titanium sticking layer, which is discussed more in the next section.

\subsection{Normal Metal Contacts}

Thanks to their application in CNTFET technologies, normal metal contacts to carbon nanotubes are the best researched of all of the materials discussed here. There are three popular choices for normal metal contacts to nanotubes. The first is palladium, which was first tested because its work function is very close to that of a typical single-walled  nanotube \cite{Javey2003, Jejurikar2010}. The second and third are Ti/Au and Cr/Au. These are common evaporated materials and, like palladium, gold is thought to make good contact to carbon nanotubes. The Ti and Cr layers are necessary for sticking layers. Titanium is difficult to thermally evaporate and had a tendency to overheat samples, melting the PMMA layer while under vacuum, for this reason it was not used. 

\begin{figure}
    \centering
    \includegraphics[width=0.6\textwidth]{chapter4/nm_cnt_contacts.png}
    \caption{Histograms showing distribution of normal metal contact resistances. Insets show the yield of devices with resistance below \SI{1}{\giga\ohm}}
    \label{fig:nm_contacts}
\end{figure}

Figure \ref{fig:nm_contacts} shows the results for the different normal metal deposition methods tested. Palladium appears to give more reliable and lower resistance contacts than Cr/Au. This is consistent with other work on the subject \cite{Javey2003}.

\subsection{Additional Comments}

Contrary to popular opinons, this work suggests that sputtering produces the most consistent, low resistance metallic contacts to carbon nanotubes. Many factors could contribute to this, such as materials choice, energy scales of the deposited materials, substrate temperatures during deposition, and contamination in the vacuum chambers. A wide range of materials were tested showing that when a device had a measurable resistance, it was consistent with existing literature. 

The results of depositions done in the sputtering chamber were relatively consistent across the different materials tested. Additionally, the rate of deposition and partial pressure in the sputtering chamber did not seem to have a noticeable effect on the resulting devices over the ranges tested. More information is available in Appendix \ref{subsec:sputtering}. 

Faster deposition rates in the thermal evaporator did show a connection to lower device resistances. This suggests that there may have been some contamination introduced in the thermal evaporation that was lessened by shortening the evaporation time. Similarly, the thermal anchoring of samples in the thermal evaporator had a marked effect on the measured resistances. Overheating of the sample in the chamber during some longer titanium evaporations was enough to warp the PMMA mask. A temperature controlled stage could eliminate this problem. Electron beam evaporation showed many of the same sample temperature control problems which were part of the reason this method was not tested further.

\section{Reducing Contamination}

\subsection{Choice of Developer}
\label{subsec:developer_choice}

Recent research has shown that the choice of developer and developer temperature can have a large effect on the resolution of and residue left by electron beam lithography \cite{Maximov2009, Macintyre2009, Aurich2012}. All electron beam lithography in this thesis used PMMA as the copolymer resist. Two different developers and two different temperatures were tested. More details on recipes and developer choice are available in Appendix \ref{chap:fabrication}

Figure \ref{fig:developer_tests} shows the distribution of carbon nanotube devices resistances obtained using each developer. Again, this data includes all materials and deposition methods.

\begin{figure}
    \centering
    \includegraphics[width=0.6\textwidth]{chapter4/developer_tests.png}
    \caption{Histograms showing contact resistance as a function of developer choice. Data includes all metals and deposition methods. Insets show the yield of devices with resistance below \SI{1}{\giga\ohm}}
    \label{fig:developer_tests}
\end{figure}

Room temperature development using a 1:3 mixture of MIBK:IPA is the most common recipe and appears to yield the most working devices. However, Low temperature development with 7:3 IPA:water appears to result in lower resistances for working devices. Since this data doesn't control for other variables, it is difficult to say which is the better choice. Development in low temperature IPA:water does give considerably better resolution. 

In addition to the nanotube device measurements, AFM measurements were used to investigate the amount of residue left behind by different developers. An example of this can be seen in Figure \ref{fig:roughness}. These roughness measurements were motivated by the inconclusive results in Figure \ref{fig:developer_tests}.

\begin{figure}
    \centering
    \includegraphics[width=0.7\textwidth]{chapter4/ipa_water_cold_roughness.png}
    \caption{Surface roughness after development of PMMA film using \SI{0}{\celsius} IPA:water in 7:3 mixture.}
    \label{fig:roughness}
\end{figure}

The roughness results for all developers tested are summarized in Table \ref{table:surface_roughness} Room temperature IPA:water roughness was not measured, but it appears to be consistent with cold IPA:water. 

\begin{table}
    \centering
    \caption{Surface roughness after development of PMMA films.}
    \begin{tabular}{ r | c c c}
        Developer & Temperature ($^{\circ}$C) & RMS Roughness (nm) & Mean Roughness (nm) \\ \hline
    7:3 IPA:water & \SI{0}{\celsius} & 0.548 & 0.436 \\
    1:3 MIBK:IPA & \SI{0}{\celsius} & 4.231 & 1.985 \\
    1:3 MIBK:IPA & \SI{23}{\celsius} & 10.871 & 8.081 \\
    \end{tabular}
    \label{table:surface_roughness}  
\end{table}

Based on the roughness measurements, IPA:water seems to be the clear choice of developer to minimize PMMA residue after development. The poor device yield seen in Figure \ref{fig:developer_tests} is likely a result of another variable not being controlled for.

\subsection{Dose Scaling}

Previous research has shown that overexposing PMMA films can lower the roughness of the surface after development \cite{Macintyre2009, Aurich2012}. Looking at Table \ref{table:surface_roughness}, the height of the PMMA residue left behind can be on the order of the nanotube diameter. In order to limit the effect this residue has on the resulting metal/nanotube contacts, it is important to reduce the amount of PMMA residue as much as possible.

There are many computationally intensive algorithms for scaling the the electron beam dose to a PMMA film based on the pattern geometry \cite{Eisenmann1993, Stirniman1994, Soe2000, Osawa2001}. Because of the high throughput of samples, and limits of our lithography software, a much simpler algorithm was developed for this work. The most important parameter in determining the dose is the smallest width of the polygon being written. This is difficult to determine because polygons in the lithography pattern are not always rectangular or aligned with the coordinate axes. A simple proxy for the smallest width across a polygon is $2\times(Area/Perimeter)$. This is easily seen for a rectangle with a high aspect ratio, $a \ll b$:

\begin{equation}
    \label{eq:dose_scaling}
    2\frac{Area}{Perimeter} = 2\frac{ab}{2a+2b} = \frac{1}{\frac{1}{a}+\frac{1}{b}} \sim a
\end{equation}

The dose must be scaled by the inverse of this, $\frac{1}{2}\times(Perimeter/Area)$ such that narrow features receive a higher dose. Once this factor is determined, empirical scaling and dose limits are needed to produce the actual dose values for the lithography pattern. An example of the results of this dose scaling can be seen in Figure \ref{fig:dose_scaling}.

\begin{figure}
    \centering
    \includegraphics[width=0.6\textwidth]{chapter4/dose_scaling.png}
    \caption{An example of a lithography pattern with dose scaling based on polygon geometry. Inset shows a close up of the nanotube contacts.}
    \label{fig:dose_scaling}
\end{figure}

Scaling the doses in this way allows for the nanotube contacts to be overexposed, reducing the PMMA residue in those areas, without overexposing the larger features connecting to bonding pads. Overexposing the larger features can lead to degradation of the developed pattern and shorts in the final device.

Implementing this geometry based dose scaling was seen to improve the yield of low resistance nanotube contacts. It also lowered the failure rate of electron beam lithography and metal liftoff by preventing shorts caused by overexposure in large features. The lowest resistance contacts with the best lithograhy resolution and device yield appear to be possible with a combination of cold IPA:water developer and the dose scaling described here.

\subsection{Annealing}

Three types of annealing were tested to improve contact resistances in carbon nanotubes. The first is high temperature annealing in an \ce{Ar}/\ce{H2} atmosphere \cite{Garcia2012, Lee2000, Kane2009, Kane2011, Stokes2010} designed to remove residue from polymer resists. Similarly, vacuum annealing at slightly higher temperatures was tested to remove water contamination and polymer resist residue \cite{Derycke2002, Kim2003, Pirkle2011, Chan2012, Cheng2011}. Finally, current annealing was tested. The motivation behind this method was to increase the current enough to cause a dielectric breakdown in any oxides inhibiting the nanotube/metal contacts \cite{Gramich2015, Wu2010}.

High temperature annealing in an \ce{Ar}/\ce{H2} atmosphere was the most successful method tested. Samples were loaded into the same tube furnace used in CVD growth and heated to \SI{325}{\celsius} while flowing \SI{500}{\sccm} Ar and \SI{500}{\sccm} \ce{H2} for 3 hours. In Cobalt and Permalloy contacts this method was found to improve contact resistances by a factor of 2-10x. While annealing did improve room temperature resistances, it did not lower noise levels, which is supported by previous work \cite{Preusche2009}. For metals requiring a titanium or chromium sticking layer, such as Nb, Al, and Au, this method did not noticeably improve contact resistances and often destroyed the contacts completely. Based on the data collected, it appears the high temperature annealing either cause, or worsened, gate leaks in the devices. It is not clear why these gate leaks appear after this anneal and not after the much higher temperature CVD growth of carbon nanotubes on the substrates. Gate leaks ultimately made this method useless for producing quantum dot devices. 

Vacuum annealing was tested on only a few devices. The method was adopted from similar work done on graphene. Samples were loaded into the tube furnace, which was evacuated to \SI{1e-6}{\torr} and heated to \SI{500}{\celsius} In the few experiments done, vacuum annealing was not seen to improve the contact resistances in any metals tested. 

Current annealing was tested on a number of devices as a last effort to get the sample working. Samples were placed in the probe station and the two contacts of interest were probed. A voltage bias was ramped repeatedly from 0 to a few volts, with a safety resistor preventing the current from exceeding \SI{25}{\micro\ampere}. This method was seen to lower contact resistances after many sweeps. This is either due to dielectric breakdown in the contacts or high temperature annealing due to Joule heating \cite{Maki2004, Woo2007, Dong2007}. Despite some success the method was extremely difficult to control. Most samples were destroyed in the process. A typical trace is seen in Figure \ref{fig:current_anneal}. Contact resistances were never successfully lowered to a useful value before the device died.

\begin{figure}
    \centering
    \includegraphics[width=0.6\textwidth]{chapter4/current_anneal.pdf}
    \caption{A typical current anneal trace. Initially, the resistance was seen to decrease. After sometime the device abruptly died.}
    \label{fig:current_anneal}
\end{figure}
    
\section{Electrical Noise in CNT Contacts}

Electronic noise in the measured current can be used as a measure of the amount of contamination in the metal/nanotube interfaces and on the substrate. The primary sources of contamination on the devices are polymer residues left by the lithography process, organic contamination from thin film deposition, and amorphous carbon deposited during the CVD growth of carbon nanotubes. Each of these contaminates affects the transport properties of the completed device. 

All of the measurements discussed in this thesis were voltage biased, DC current measurements. For each data point, the applied voltages ($V_{gate}$, $V_{bias}$) were set, and the current sampled at a fixed rate for a specified amount of time. Typically, data was taken at 3-\SI{40}{\kilo\hertz} for 0.1-\SI{0.5}{\second}. In many cases, the entire time series at each point was dumped into a binary data file for later analysis. These current versus time measurements at fixed gate and bias voltages provide the data used in the following noise analysis.

A simulated current noise power spectrum is plotted in Figure \ref{fig:simulated_noise}. At low frequencies, the noise is dominated by the $1/f$ contribution. At higher frequencies the $1/f$ becomes insignificant and the noise level is dominated by shot noise, which is attributable to the discrete nature of charge moving through the device. Over all frequency ranges, there is a constant contribution from Johnson thermal noise.

\begin{figure}
    \centering
    \includegraphics[width=0.6\textwidth]{chapter4/noise_types.png}
    \caption{Simulated noise power spectrum for a CNTFET.}
    \label{fig:simulated_noise}
\end{figure}

\subsection{Random Telegraph Noise}
\label{sec:RTN}

Random telegraph noise in electronic transport measurements is caused by the hopping of charge on and off of impurity charge traps capacitively coupled to the device. In a nanotube quantum dot, these impurities are most likely in one of three places; at the nanotube-metal interface \cite{Liu 2006}, at the nanotube-substrate interface \cite{Liu2006, Sydoruk2014}, or traps can be located at defects in the nanotube itself \cite{Stokes2010}.

Random telegraph noise manifests in the time series data as square pulses in the measured current with random durations. The heigh of each pulse can be used to identify the pulses as being caused by one or more impurities with different charging energies.

\begin{figure}
    \centering
    \includegraphics[width=0.8\textwidth]{chapter4/diamond_noise.png}
    \caption{Noise power spectra from a nanotube device at \SI{4}{\kelvin}. $V_{gate}$ is fixed. Each curve is taken at a different $V_{bias}.$.}
    \label{fig:diamond_noise}
\end{figure}

Looking at the conductance as a function of $V_{bias}$ and $V_{gate}$ voltage in Figure \ref{fig:diamond_noise}, two types of random telegraph noise are apparent, with distinctly different time scales. Consider first the switching behavior seen as a function of the gate voltage near $V_{gate} = 0.36V$ in Figure \ref{fig:diamond_noise}. This behavior can be explained by charge trapping in impurities at the gate oxide/nanotube interface. As the gate voltage is swept charge can be trapped and released from impurities capacitively coupled to the nanotube. These impurity charge traps then act as local gates which shift the Coulomb diamond plot abruptly to the left or right. This can be thought of as a random telegraph signal with a long time constant, on the order of tens of minutes at low temperatures. At room temperature these trapping events are much more frequent leading to a hysteresis behavior in the gate sweep, rather than a discrete charging/discharging event. Figure \ref{fig:gate_hyst} shows an example of this gate hysteresis caused by impurities capacitively coupling to the nanotube.

\begin{figure}
    \centering
    \includegraphics[width=0.6\textwidth]{chapter4/gate_hyst.png}
    \caption{Hysteresis in a CNTFET $I$-$V_{gate}$ curve.}
    \label{fig:gate_hyst}
\end{figure}

Near $V_{gate} = 0.3V$ in Figure \ref{fig:diamond_noise}, random telegraph noise with much shorter characteristic time scales is observed. This becomes more clear by plotting several time series as a function of $V_{bias}$ at a fixed $V_{gate}$, as seen in Figure \ref{fig:rts_bias}.

\begin{figure}
    \centering
    \includegraphics[width=0.9\textwidth]{chapter4/random_telegraph_noise.png}
    \caption{Left: Measured current as a function of $V_{bias}$ at $V_{gate} = 0.304V$. Right: Time series at fixed $V_{bias}$ and $V_{gate}$. Colors for each individual time series correspond to those in the bias sweep.}
    \label{fig:rts_bias}
\end{figure}

The time series data in the right panel of Figure \ref{fig:rts_bias} clearly shows the random telegraph noise in the current measurement. The distribution of switching times has not been carefully measured, but appears to range from about 10-\SI{1000}{\milli\second}. Because the current is only sampled for 0.1-0.5s in a typical measurement (0.4s in the sample being discussed here), it is difficult to get accurate distributions of switching times. Some useful information can be gathered from the fact that there appears to be only two levels in the system. That means the random telegraph noise signal originates from charge tunneling to and from a single defect, most likely at one of the metal/CNT contacts. 

\subsection{$1/f$ Noise}

$1/f$ noise, or pink noise, is the dominate noise source in nearly every electronic device at low frequencies. Measurements of $1/f$ noise amplitude and how it scales with applied bias and fields can offer insight into the origins of noise and disorder in a semiconductor device. Because the devices made for this thesis had such persistant problems with noise and contamination, it was thought that investigating the $1/f$ noise data could illuminate the source of some of the noise.

The first work on $1/f$ noise was published in 2000 \cite{Collins2000}. This, and additional work in 2006 \cite{Ishigami2006}, confirmed that carbon nanotube field effect transistors operating at room temperature have a $1/f$ noise amplitude that follows the empirical model proposed by Hooge in 1969 \cite{Hooge1969}. This model predicts that the noise amplitude in a semiconductor device should be proportional to $I^2$ and inversely proportional to $f$:

\begin{equation}
    \label{eq:hooge}
    S_{I} = A\frac{I^2}{f}
\end{equation}

Later results \cite{Tobias2008} showed that this model holds down to $1.2K$ for relatively long channel lengths, $L=3\mu m$. Only a small amount of research exists on the behavior of $1/f$ noise when electron confinement becomes important at low temperatures. The most significant work was done on \ce{GaAs} quantum dots \cite{Jung2004} and later repeated for suspended graphene quantum dots \cite{Song2015}. Similar to those two works, all data presented here was taken at \SI{4}{\kelvin} at a range of bias and gate voltages in the Coulomb blockade regime. A sample of the raw data can be seen in Figure \ref{fig:1-f_data}.

\begin{figure}
    \centering
    \includegraphics[width=0.6\textwidth]{chapter4/1-f_real_data.png}
    \caption{Noise power spectra from a nanotube device at \SI{4}{\kelvin}. $V_{gate}$ is fixed. Each curve is taken at a different $V_{bias}.$.}
    \label{fig:1-f_data}
\end{figure}

The data in Figure \ref{fig:1-f_data} does fit the $1/f$ dependence well. To test if this data fit with Hooge's law, the left side of Figure \ref{fig:noise_current} shows the average value of $S_I \cdot f$ plotted against the measured current (each measured current corresponds to a single applied bias voltage) for a range of gate voltage values annotated in the plot legend. It is immediately clear from this plot that the current does not scale as $I^2$. Much of the data shows a plateau in $S_I \cdot f$ near resonances in the Coulomb diamond diagram, and an abrupt drop in noise off resonance. The fact that the noise amplitude does not follow Hooge's law suggests that the $1/f$ noise is not dominated by noise from the carbon nanotube itself. Instead, the noise must come from another source, most likely from charge trapping in impurity and quantum dot levels. Additional experiments might check the temperature and frequency dependence of this behavior to determine the exact mechanism.

The plot on the right in Figure \ref{fig:noise_current} shows $S_I \cdot f$ as a function of the measured conductance. This data does not show any obvious relation between the two quantities. 

\begin{figure}
    \centering
    \includegraphics[width=0.9\textwidth]{chapter4/noise_amplitude_scaling.pdf}
    \caption{Left: $S_I \cdot f$ plotted against measured current with $V_{bias}>0$ at a range of $V_{gate}$. Right: The same data plotted against conductance.}
    \label{fig:noise_current}
\end{figure}

In order to get a better idea of this noise plateau near resonances in the Coulomb diamond conductance plot, the two are compared in Figure \ref{fig:noise_spectroscopy}. The noise amplitude plot at the bottom of Figure \ref{fig:noise_spectroscopy} confirms that there are maxima in the $S_I \cdot f$ data that correspond to maxima in the conductance data, seen in the top plot. Additionally, the noise amplitude plot shows increases in $1/f$ noise in places where there is a lot of random telegraph noise, such as near $V_{gate} = -7V$ and $V_{bias}<0$. This increased $1/f$ noise near regions with excess random telegraph noise may suggest an explanation of the overall behavior. 

\begin{figure}
    \centering
    \includegraphics[width=0.9\textwidth]{chapter4/noise_spectroscopy.pdf}
    \caption{Conductance (Top) and $\log_{10}{S_I \cdot f}$ (Bottom) as a function of $V_{bias}$ and $V_{gate}$.}
    \label{fig:noise_spectroscopy}
\end{figure}

The $1/f$ noise increases when there are impurity charge traps that can be accessed by the electrons on or near the quantum dot. For these low dot fillings and low temperatures, the quantum dot itself behaves like a charge trap, where the $1/f$ noise is increased when the Fermi energy on the source/drain are near a resonant level because of discrete tunneling events on and off the quantum dot. Thus, in this regime the $1/f$ noise amplitude can be used as a spectroscopic probe into the energy levels on the quantum dot.

In Section \ref{sec:RTN}, the characteristic times for charge hopping onto impurities was found to be on the order of 10-\SI{1000}{\milli\second}. This corresponds to a frequency range of 1-\SI{100}{\hertz}. Those frequencies are well within the range considered in calculating the $1/f$ noise amplitude. Therefore, it should be no surprise that the noise amplitude increases in regions showing excess random telegraph noise. This also suggests that the characteristic times for electrons tunneling onto the quantum dot levels fall into a similar range of 0.1-\SI{1000}{\milli\second}. This range of characteristic tunneling times is consistent with previous measurements of the tunneling rate onto a nanotube quantum dot \cite{Gotz2008}. 

In previous measurements, the tunneling rates were tuned by varying the barrier heights with local gating. Here, the fluctuations in barrier height are not controlled, the quantum dot is in a static configuration. Therefore, the fluctuations in potential are related to impurity charge trapping in the tunnel barriers or gate oxide. This conclusion is supported by previous research into barrier height fluctuations \cite{Jung2004} and dependence of $1/f$ noise on charge trapping in different gate oxide materials \cite{Sydoruk2014}. It was suggested in Reference \cite{Sydoruk2014} that gamma irradiation of quantum dot samples may eliminate some of these charge trapping problems, which may be an interesting avenue for future work.

\subsection{Multiple Quantum Dots}

One final means of finding noise sources in the sample comes from the measurement of overlapping Coulomb diamonds in conductance plots. An example of this can be seen in Figure \ref{fig:overlapping_diamonds}.

\begin{figure}
    \centering
    \includegraphics[width=1.0\textwidth]{chapter4/overlapping_diamonds.pdf}
    \caption{Two different Coulomb diamond scales visible in the conductance plots. Top: A close view showing the islands corresponding to the device size. Bottom: Expanded view showing diamonds corresponding to a much smaller quantum dot.}
    \label{fig:overlapping_diamonds}
\end{figure}

An estimate of the two quantum dot sizes seen in Figure \ref{fig:overlapping_diamonds} can be obtained using simple relations discussed in Chapter \ref{sec:CNT} \cite{Bockrath1997}. The top plot gives the following parameters, $\Delta E \sim 2.5meV$ and $\Delta \mu \sim 7meV$. There is no four-fold degeneracy visible in the diamond sizes, which suggests $E_{charging} \gg \Delta E$. With that approximation, the dot size can be estimated in two ways, seen in Equation \ref{eq:dot_sizes}.

\begin{align}
    \Delta E \sim \frac{0.5eV}{L(nm)} = 2.5meV \nonumber \\
    E_{charging} = \frac{e^2}{C} \sim \frac{e^2}{L} = \frac{1.4eV}{L(nm)} = 7meV
    \label{eq:dot_sizes}
\end{align}

Both of these relations gives a quantum dot size of $L\sim200nm$, which is roughly the size of the device as it was designed. 

The larger dots seen in the image are a little more difficult to understand. A quick estimate can be obtained in the same way as Equation \ref{eq:dot_sizes}. Looking at the size of the diamond highlighted in black gives a rough estimate of $L \lesssim 100nm$ which is less than half the size of the device as it was designed. This suggests some contamination on the substrate or defect is acting like a quantum dot in series with the intended one. Similar measurements have been made in the past \cite{Stokes2010, McEuen1999, Park2001, Babi2003} in which the additional quantum dots were attributed to defects on the substrate; that explanation is consistent with these observations.