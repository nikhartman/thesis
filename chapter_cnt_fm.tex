\chapter{Quantum Dots with Ferromagnetic Leads}
\label{sec:FMCNTQD}
\chaptermark{Ferromagnetic CNTQD}

Of the many samples discussed in Chapter \ref{chap:contacts}, seven F-CNT-F devices were measured at low temperatures. Details, including room temperature resistance, can be seen in Table \ref{table:rt_fm_devices}.

\begin{table}
    \centering
    \footnotesize
    \begin{tabular}{ r | c | c c c}
        Sample & Leads & Material & Deposition Method & R($T=300K$) \\ \hline
        SCF72 & 17-19 & Co & sputter & \SI{65}{\kilo\ohm} \\
        SCF72 & 21-23 & Co & sputter & \SI{76}{\kilo\ohm}\\
        SCF75 & 21-23 & Co & sputter & \SI{280}{\kilo\ohm}\\
        SCF75 & 15-16 & Co & sputter & \SI{30}{\kilo\ohm}\\
        SCF96 & 9-12  & Co & electron beam evaporation & \SI{200}{\kilo\ohm}\\
        SCF96 & 16-17 & Co & electron beam evaporation & \SI{400}{\kilo\ohm}\\
        SCF98 & 11-12 & Co & electron beam evaporation & \SI{120}{\kilo\ohm}\\
        \label{table:rt_fm_devices}  
    \end{tabular}
    \caption{Details of measured F-CNT-F devices}
\end{table}

Each of these samples was measured at 4K, with the exception of SCF98 (11-12) which was measured at 150mK. Figure \ref{fig:all_FM_QD} shows Coulomb diamond plots for each of the samples. These plots were used to calculate the quantum dot properties seen in Table \ref{table:cold_fm_devices}

\begin{table}
    \centering
    \footnotesize
    \begin{tabular}{ r | r | c c c c c c c}
        Sample & Leads & $\Delta \mu (meV)$ & $\Delta E (meV)$ & $\frac{e^2}{2C} (meV)$ & $L_{measured} (nm)$ & $L_{design}$ (nm) & $C_{\Sigma} (aF)$ & $\alpha_{G}$ \\ \hline
        SCF72 & 23-21 & 8  & x   & x    & 175 & 300 & x  & 0.080 \\ 
        SCF72 & 17-19 & 8  & 1.8 & 6.2  & 280 & 300 & 13 & 0.072 \\
        SCF75 & 21-23 & 15 & 4.5 & 11.5 & 110 & 300 & 7  & 0.075 \\
        SCF75 & 15-16 & 12 & 3.5 & 8.5  & 140 & 300 & 9  & 0.110 \\
        SCF96 & 9-12  & 30 & 4.0 & 26.0 & 125 & 300 & 3  & 0.011 \\
        SCF96 & 16-17 & 22 & 2.0 & 20.0 & 250 & 300 & 4  & 0.006 \\
        SCF98 & 11-12 & 7  & 1.8 & 5.2  & 280 & 300 & 15 & 0.005 \\
        \label{table:cold_fm_devices}  
    \end{tabular}
    \caption{Low temperature characteristics of F-CNT-F quantum dots}
\end{table}

\begin{figure}
    \centering
    \includegraphics[width=1.0\textwidth]{fmdots/all_fm_dots.png}
    \caption{Conductance as a function of $V_{bias}$ and $V_{gate}$ for all measured ferromagnetic quantum dots.}
    \label{fig:all_FM_QD}
\end{figure}

The lever arm, $\alpha_{G}$, defined as $\alpha_{G} = C_G/C_{\Sigma}$ and is used to convert gate voltages to energies, $U_{gate} = \alpha_{G}eV_{gate}$ \cite{Ihn2004}. It is calculated based on the geometry of the Coulomb diamonds, as illustrated in Figure \ref{fig:alpha_calc} and characterizes the coupling regime for each device. The electron beam evaporated samples are in a much weaker coupling regime than the sputtered samples. All other aspects of the fabrication were the same over all four chips. 

\begin{figure}
    \centering
    \includegraphics[width=0.6\textwidth]{fmdots/alpha_calc.pdf}
    \caption{Calculating the lever arm using Coulomb diamond geometry.}
    \label{fig:alpha_calc}
\end{figure}

\section{Tunneling Magnetoresistance}
\label{sec:TMR}

Tunneling magnetoresistance is a change in resistance measured across a ferromagnet/insulator/ferromagnet structure as a function of magnetic field \cite{Julliere1975}. The spin transport through such an interface can be made tunable by replacing the insulator with a tunable quantum dot. Carbon nanotubes quantum dots are an attractive choice due to their well-understood spin and orbital filling. Exchange coupling with the ferromagnetic leads will split the spin degeneracy of the quantum dot, making it possible to tune the magnitude and even the sign of the tunneling magnetoresistance \cite{Tsymbal2003, Sahoo2005, Thamankar2006}.

The magnetization of the carbon nanotube contacts is dominated by their shape anisotropy \cite{Morrish1983}. This is typical for polycrystaline sputtered and electron beam evaporated films, such as the ones used in these samples \cite{Johnson1999}. For long, narrow, rectangular ferromagnet films, the shape anisotropy depends primarily on the width of the film. Wider films have slightly lower coercive fields at which the magnetization will be flipped as an external, parallel magnetic field is varied. By patterning two ferromagnetic contacts of different widths on a carbon nanotube, tunneling magnetoresistance can be measured through the nanotube quantum dot. A schematic of the situation can be seen in Figure \ref{fig:spin_valve}.

\begin{figure}
    \centering
    \includegraphics[width=0.9\textwidth]{fmdots/TMR.pdf}
    \caption{Top: A diagram of the magnetization of the ferromagnetic nanotube contacts corresponding to the data shown in the lower plot. Bottom: TMR signal measured through a carbon nanotube quantum dot.}
    \label{fig:spin_valve}
\end{figure}

\subsection{Results}

Of the ferromagnetic devices measured, only sample SCF72 showed clear tunneling magnetoresistance behavior. In the other devices the effect was overwhelmed by random telegraph noise in the data, which can look like false TMR signals, or metal/nanotube barriers with impurities causing spin-flip events that destroyed the TMR effect.

\begin{figure}
    \centering
    \includegraphics[width=0.9\textwidth]{fmdots/scf72_17-19_TMR_4K.pdf}
    \caption{Tunneling magnetoresistance measurements taken on device SCF72 (17-19) at $V_{gate} = 0V$. The plots show data for $V_{bias} = 5mV$ (left) and $V_{bias} = 10mV$ (right).}
    \label{fig:TMR_real}
\end{figure}

In the simplest model, tunneling magnetoresistance is defined as:

\begin{align}
    TMR &\equiv \frac{G_P - G_{AP}}{G_P + G_{AP}} = \frac{R_{AP} - R_{P}}{R_P + R_{AP}} \\
    TMR &= P_L P_R \label{eq:basic_tmr}
\end{align}

Where $P_{L(R)}$ are the spin polarizations at the Fermi level for the left and right nanotube contacts \cite{Maekawa1982}. This polarization is defined as:

\begin{equation}
    P_{L(R)} = \frac{\rho^{L(R)}_{+} - \rho^{L(R)}_{-}}{\rho^{L(R)}_{+} + \rho^{L(R)}_{-}}
\end{equation}

Where $\rho^{L(R)}_{+(-)}$ are the tunneling density of states for the majority (minority) spins at the Fermi level. In this picture, $P_{L(R)}$ will be positive, thus the TMR signal will be positive.

In the data shown in Figure \ref{fig:TMR_real}, the TMR signal is measured to be about -6.0\% and -0.8\% in the left and right plots, respectively. Additionally, the coercive fields for the two cobalt contacts are 40mT and 60mT. These coercive field measurements are consistent with the predicted values for narrow cobalt wires.

The TMR signal has a magnitude consistent with the predicted values \cite{Maekawa1982}, but with the opposite sign. This change in sign of the TMR signal has previously been explained by a resonant tunneling model with asymmetric coupling of the left and right contacts to the nanotube. The model as described here is taken from \cite{Tsymbal2003}.

First the conductance through the quantum dot can be written as:

\begin{equation}
\label{eq:tunnel_conductance}
    G(E) = \frac{4e^2}{h} \frac{\Gamma_L \Gamma_R}{(E-E_i)^2 + (\Gamma_L + \Gamma_R)^2}
\end{equation}

Here, $\Gamma_{L(R)}$ are the tunneling rates on/off the quantum dot from the left(right) ferromagnetic contacts. These tunneling rates can be thought of as being proportional to the density of states of the on the leads. $E_i$ is the energy of the quantum dot level nearest to the Fermi level of the leads, $E$. 

When the quantum dot is tuned off resonance, $|E-E_i| \gg \Gamma_L + \Gamma_R$, the TMR signal is predicted by Equation \ref{eq:basic_tmr}. Near a resonant level, $E \sim E_i$ the situation is quite different. Considering the limit where $\Gamma_L \gg \Gamma_R$, Equation \ref{eq:tunnel_conductance} simplifies to $G \sim \frac{\Gamma_R}{\Gamma_L}$. The conductance is now inversely proportional to the density of states on one of the leads, which leads to a sign inversion of the TMR signal.

\begin{equation}
\label{eq:sign_inversion}
    TMR = -P_L P_R
\end{equation}

By using a carbon nanotube quantum dot, the TMR signal can be modulated by tunning the resonant energy level $E_i$ with the gate voltage. Additionally, the magnitude can be modulated because the levels on the quantum dot are spin non-degenerate due to the exchange coupling field introduced by the magnetic leads. This effect has been observed previously in carbon nanotube quantum dots \cite{Sahoo2005, Thamankar2006}. 

In sample SCF72, the TMR signal was only significant in a narrow range of gate voltages near $V_{gate}=0V$, as shown in Figure \ref{fig:TMR_real}. Looking at the conductance data as a function of the gate voltage in Figure \ref{fig:TMR_gate}, it is clear that at $V_{gate}=0V$ there is a quantum dot level $E_i$ resonant with the Fermi level on the leads. This TMR measurement is used in Section \ref{sec:spin_selection_field} to establish that tunneling onto the dot conserves spin, and tunnel rates from the source and drain are asymmetric.

\begin{figure}
    \centering
    \includegraphics[width=0.5\textwidth]{fmdots/scf72_17-19_gateswp-for-TMR.pdf}
    \caption{Conductance as a function of $V_{gate}$ for sample SCF72. Note the conductance peak at $V_{gate}=0V$.}
    \label{fig:TMR_gate}
\end{figure}

\section{Negative Differential Conductance in Coulomb Blockade Regime}

Negative differential conductance resulting from spin blockade has been observed in two devices measured, seen in Figure \ref{fig:negative_differential_conductance}. These measurements will be compared to a previous measurement of a carbon nanotube double quantum dot \cite{Buitelaar2008} as well as a theoretical model predicting spin suppression in single quantum dots \cite{Weinmann1994}. This work is the first instance of these effects being observed in a F-CNT-F device. Neither device showed clear TMR measurements, but we assume there is some spin transport through the tunnel barriers, as well as asymmetry in the source/drain tunneling rates based on the contact widths.

\begin{figure}
    \centering
    \includegraphics[width=0.9\textwidth]{fmdots/scf75_scf96_ndc.pdf}
    \caption{Conductance as a function of $V_{bias}$ and $V_{gate}$ in samples SCF75 (left) and SCF96 (right). Note the differential conductance peaks in dark blue running parallel to the diamond edges in each sample.}
    \label{fig:negative_differential_conductance}
\end{figure}

\subsection{Spin Blockade in a Double Quantum Dot}
\label{sec:dqd_model}

Defects or impurities along the length of a nanotube quantum dot can form small tunnel barriers that interrupt electron transport through the dot \cite{McEuen1999, Bockrath2001}. These tunnel barriers define multiple quantum dots between the fabricated leads on the sample. This is a commonly observed problem in substrate grown carbon nanotubes. Other work has exploited similar defects to form tunable double quantum dots \cite{Mason2004, Jorgensen2007}.

With careful analysis, the formation of multiple quantum dots in series can lead to interesting physical situations not observed in single quantum dot devices. Two quantum dots formed in series by a defect or impurity have much lower tunnel barriers and stronger coupling than double quantum dots formed by the patterning of local gates. Thanks to this strong coupling between dots, spin interactions can be observed that are not seen in as-fabricated double quantum dot structures. In 2008, Buitelaar et al. used this idea to measure Pauli spin blockade in a carbon nanotube double quantum dot \cite{Buitelaar2008}. The model relies on the formation of a double quantum dot by an impurity.

Table \ref{table:cold_fm_devices} shows samples SCF75 (15-16) and SCF96 (9-12) have calculated lengths of 140nm and 110nm, respectively. Comparing this to the 300nm spacing between cobalt leads suggests that the quantum dot being measured in each case is defined by defects along the nanotube. This is supported by the gate measurements that show no clear periodicity in Figure \ref{fig:scf75_scf96_gates}. One final piece of evidence for a double quantum dot is the suppressed conductance in the SCF75 levels around $V_{gate}=5.9V$, this can be attributed to one dot being off resonance while the other is not. This feature also suggests the back gate coupling for each quantum dot is quite different, thus the gate dependent features are dominated by a single dot in the series.

\begin{figure}
    \centering
    \includegraphics[width=0.9\textwidth]{fmdots/scf75_scf96_gates.pdf}
    \caption{Current as a function of $V_{gate}$ in samples SCF75 (top) and SCF96 (bottom). No clear periodicity is visible in either measurement implying disorder along the nanotube.}
    \label{fig:scf75_scf96_gates}
\end{figure}

Figure \ref{fig:buitelaar_levels} illustrates the basic principles of the proposed model. It based on a hybridization between levels in the two strongly coupled quantum dots. Spin blockade then comes from a forbidden transition between the (1,1) spin triplet state and the (0,2) spin singlet state.

\begin{figure}
    \centering
    \includegraphics[width=1.0\textwidth]{fmdots/buitelaar_levels.png}
    \caption{Figure adapted from \cite{Buitelaar2008}. (a) Double quantum dot levels. Numbering refers to the occupation of (QDA,QDB). $\delta \epsilon$ is the splitting between single particle energy levels. $U$ and $U'$ are the charging energies. (b) With strong coupling the (1,1) and (0,2) states hybridize to form bonding and anti-bonding orbitals $S_1$ and $S_2$ different from the triplet state $T$. (c) At negative bias the (1,1) to (0,2) transition is forbidden by spin blockade as the (0,2) must be a spin singlet. (d) The transition is allowed in the reverse bias situation.}
    \label{fig:buitelaar_levels}
\end{figure}

Figure \ref{fig:dqd_model_compare} shows the data from SCF75 alongside the simulation from Buitelaar et al. Note the similarities in the negative differential conductance peak in the data and simulation. The simulation was run for a device 300nm in length with a defect in the center, giving two dots of size 150nm, consistent with Table \ref{table:cold_fm_devices}. Using $\alpha_G = 0.11$ , gives an energy spacing between the two zero bias conductance peaks of 15.4meV, which agrees well with the simulation parameters. The situation is identical for SCF96, with the exception of the gate coupling.

\begin{figure}
    \centering
    \includegraphics[width=0.9\textwidth]{fmdots/dqd_model_compare.png}
    \caption{A comparison of the SCF75 negative differential conductance data (a) with double quantum dot data (b) and simulations (c) from Reference \cite{Buitelaar2008}.}
    \label{fig:dqd_model_compare}
\end{figure}

\subsection{NDC from Spin Selection Rules}
\label{sec:single_dot_model}

The negative differential conductance peaks seen in Figure \ref{fig:negative_differential_conductance} may be explained by a simpler, single dot model, but not necessarily captured by the constant interaction model discussed in Section \ref{sec:constant_interaction_model}. As electrons fill the quantum dot, they are not only subject to charge interactions, but also spin interactions. Each quantum dot occupation number has many possible collective spin states. Figure \ref{fig:spin_states} tabulates some of these states for low occupation numbers, calculated from Clebsh-Gordan coefficients \cite{Weinmann1994}.

\begin{figure}
    \centering
    \includegraphics[width=0.8\textwidth]{fmdots/spin_states.png}
    \caption{Table tabulating possible occupation numbers, spin states, and relevant energy level splittings for low occupation number quantum dots as calculated by Weinmann et al. \cite{Weinmann1994}.}
    \label{fig:spin_states}
\end{figure}

This means that there are spin selection rules that can determine the lifetime of a given quantum dot state. Consider, for example, a state with occupation number $n$ and maximal spin, $S=n/2$. This is possible when transport occurs though excited states in a 1D quantum dot. The $S=n/2$ state must transition to a nearby state ($n-1$, $S'=(n-1)/2$). Since the spin can only change by -1/2, not $\pm$1/2, $S=n/2$ is a relatively long lived state. When this state is brought into resonance with the Fermi level on the leads, the current through the dot is reduced, leading to a negative differential conductance peak. Figure \ref{fig:weinmann_compare} compares qualitatively a simulation of this effect with the data from SCF96 \cite{Weinmann1995}. 

\begin{figure}
    \centering
    \includegraphics[width=0.8\textwidth]{fmdots/weinmann_compare.pdf}
    \caption{A comparison between simulated transport through a quantum dot with spin selection rules and data from SCF75. The simulation is taken from Weinmann et al. \cite{Weinmann1995}. Simulation data shows regions of negative differential conductance as bright white lines. Data on the right shows regions of negative differential conductance in dark blue.}
    \label{fig:weinmann_compare}
\end{figure}

A similar effect can explain the missing ground state transitions seen in the data for SCF96. If the spin ground states for two adjacent quantum dot occupation numbers differ by more than $1/2$ the conductance peak is missing from the stability diagram.

These calculations assume spin degenerate levels and symmetric coupling to normal metal leads. Without a direct measure of the exchange coupling in ferromagnetically contacted nanotubes, it is unclear if the first criterion is met. The second condition is clearly not met, but it is not required, just a simplification. It is easy to imagine that these types of spin selection rules will be exaggerated further by reducing one spin population and introducing spin dependent tunnel barriers. See Figure \ref{fig:scf72_info}(a) for a model of such a quantum dot.

\section{Spin Selection Rules with Applied Magnetic Field}
\label{sec:spin_selection_field}

Additional data supporting the suppression of conductance peaks in ferromagnetic quantum dots due to spin selection rules is observed in sample SCF72. This dot has been used throughout this thesis as the canonical example of a Coulomb blockade measurement. It shows regular oscillations in the gate, suggesting a single dot with levels that are not split by exchange coupling or disorder.  From the TMR measurement it is clear that tunneling onto the device preserves spin and there is an asymmetry in the source/drain tunneling rates, as seen in Figure \ref{fig:scf72_info}(a).  

\begin{figure}
    \centering
    \includegraphics[width=1.0\textwidth]{fmdots/scf72_info.pdf}
    \caption{(a) Energy level diagram for an F-CNT-F quantum dot. The two barrier colors represent the two spin dependent tunnel barrier heights for spin up and spin down electrons. Blue and red arrows illustrate the different tunneling rates for up and down spins. (b) SCF72 conductance at $B=0$. (c) SCF72 conductance at $B=4T$.}
    \label{fig:scf72_info}
\end{figure}

Figure \ref{fig:scf72_spin_selection} shows Coulomb diamond data for SCF72 at $B=0$T and -4T; suppressed conductance levels are marked with dashed lines. Conductance peaks that mark ground state transitions between different dot occupation numbers (those that cross $V_{bias}=0mV$) are suppressed as the applied field is increased. This can be understood in light of the discussion in Section \ref{sec:single_dot_model}. Applying a strong magnetic field changes the spin ground state to one in which adjacent occupation numbers have spin levels differing by more than $\Delta S = 1/2$ as predicted by the constant interaction model. Looking at the excited states, it is clear that there is a similar change in which excited states are suppressed based on the change in lowest energy spin states.

\begin{figure}
    \centering
    \includegraphics[width=1.0\textwidth]{fmdots/scf72_spin_selection.pdf}
    \caption{Conductance as a function of $V_{gate}$ ang $V_{bias}$ for sample SCF72 at $B=0$T (a) and -4T (b). Solid lines trace conductance peaks. Dashed lines trace suppressed peaks. (c) Schematic of energies on the dot. (d) Sketch of energy levels as seen in the Coulomb diamond plots.}
    \label{fig:scf72_spin_selection}
\end{figure}

The Zeeman splitting can be measured by measuring the change in excited state energies as the magnitude of the field is increased. At zero field, excited states are separated by $\Delta E = 1.8$meV. In a field of -4T, the levels are separated by $\Delta E = 2.0meV$.  This is consistent with the Zeeman splitting energies previously measured in carbon nanotube quantum dots \cite{Cobden1998, Lai2014}. This measurement does not show any splitting of degenerate levels. This suggests that the levels were not degenerate, or that some of the split levels do not participate in the conduction due to the same spin selection rules being discussed here.

\subsection{Results and Discussion}

Effects of spin selection rules have been observed in three samples. At least one of these samples, SCF72, is proven to be a clean, single quantum dot. This suggests that spin selection rules beyond the constant interaction model can be used to explain suppressed and negative differential conductance features in the Coulomb diamond data. Additional work must be done in modeling transport though a F-CNT-F dot to confirm that the features observed here can be fit using such a model. 

\section{Spectroscopy of a Magnetic Impurity}
\label{sec:imurity_tunneling}

Coupling a nanoparticle to a carbon nanotube quantum dot makes it possible, through transport spectroscopy, to measure physical properties of nanoscale particles too small to contact using typical lithographic methods. Sample SCF96 shows such coupling to a magnetic impurity. The transport through the impurity was found to be tunable by varying the gate voltage and external magnetic field. The data are shown in Figure \ref{fig:resonant_tunneling}

\begin{figure}
    \centering
    \includegraphics[width=0.9\textwidth]{fmdots/scf96_resonant_tunneling_conductance.pdf}
    \caption{Top: Conductance as a function of $V_{bias}$ and $V_{gate}$ in zero field. Bottom: Conductance measured over the same portion of the quantum dot spectrum in a 2T magnetic field.}
    \label{fig:resonant_tunneling}
\end{figure}

Looking at the data it is clear that their are positive and negative differential conductance peaks that cut through the Coulomb blockaded regions of the conductance plot at positive bias in zero magnetic field. Both the positive and negative conductance peaks are suppressed at 2T.

\subsection{Characteristic Size and Level Spacing}

Figure \ref{fig:resonance_cuts} shows cuts through each of the three Coulomb diamonds where the resonance peaks are observed. Looking at plots at constant gate voltage, it is clear that the resonances have both positive and negative peaks in the conductance, with some background conductance superimposed due to the conductance through the nanotube quantum dot levels.

\begin{figure}
    \centering
    \includegraphics[width=0.9\textwidth]{fmdots/scf96_resonant_tunneling_cuts.pdf}
    \caption{Top: Conductance as a function of $V_{bias}$ and $V_{gate}$ in zero field. Bottom: Conductance measured over the same portion of the quantum dot spectrum in a 2T magnetic field.}
    \label{fig:resonance_cuts}
\end{figure}

From the data in Figure \ref{fig:resonance_cuts} it is possible to derive the relevant energy scales of the impurity level. Positive and negative conductance peaks are split by 2meV. Adjacent positive peaks are separated by 5meV. This suggests that the impurity has at least 3 energy levels that are separated by 5meV. These three levels are spin split by 2meV, most likely because the impurity is ferromagnetic.

The charging energy for the impurity can be measured using the lever arm calculated in Table \ref{table:cold_fm_devices}. Resonant tunneling through the impurity is observed for gate voltages between 10 and 20V. With the lever arm of $\alpha_G = 0.006$ the charging energy can be estimated to be roughly 100meV, which corresponds to a capacitance of about $C_{\Sigma} = 0.8$aF. 

The most likely origin of this impurity is a piece of magnetic contamination. The size of the impurity can be estimated in two ways. First, it can be estimated using the capacitance calculated above. Considering the impurity as a spherical metal shell gives $C_{\Sigma} \sim 4\pi \epsilon_0 r = 0.8$aF. With that, the radius is calculated to be $r = 7$nm. The size can also be estimated quantum mechanically from the level spacing, $\Delta E = 5$meV. Assuming the impurity is a 3D spherical well, the levels are $E_{n,l} = \frac{\hbar^2}{2ma^2}z_{n,l}^2$ where $z_{n,l}$ is the nth zero of the lth spherical Bessel function. For $l=0$, the level spacing is $\Delta E = \frac{\pi \hbar^2}{2mr^2}$. Using the measured value $\Delta E = 5meV$ gives an estimate for the radius of $r = 5$nm. These two estimates are in good agreement and suggest an origin for the impurity. A 10nm diameter magnetic particle on the substrate perfectly describes the iron nanoparticles used in the nanotube catalyst.

\subsection{Effects on CNT Quantum Dot Levels}

Additional interactions between the dot and impurity levels can be seen at negative bias in the 2T magnetic field data. This region is highlighted in Figure \ref{fig:scf96_ndc_impurity}.

\begin{figure}
    \centering
    \includegraphics[width=0.6\textwidth]{fmdots/scf96_ndc_impurity.pdf}
    \caption{Conductance as a function of $V_{bias}$ and $V_{gate}$ at $B=2T$ in sample SCF96. Note the difference in slope in the excited states of the two Coulomb diamonds seen at negative bias and the strong negative differential conductance.}
    \label{fig:scf96_ndc_impurity}
\end{figure}

There are two features to note in Figure \ref{fig:scf96_ndc_impurity}. The first is the change in slope of the two Coulomb diamonds seen at negative bias. This is a clear indication that the conductance measured is not due to a single quantum dot as described by the constant interaction model. Additionally, note the strong negative differential conductance seen in these same diamonds. Both of these features suggest some coupling between the levels on the nanotube quantum dot and in the nearby magnetic impurity. A similar coupling has been observed previously between a magnetic impurity and vibrational modes in a suspended nanotube \cite{Ganzhorn2013}. The exact nature of the coupling is not known, but it is likely that both of these observed features can be attributed to the presence of the magnetic impurity. The negative differential conductance in particular may fit with some existing models of hybridized levels in two quantum dots resulting negative differential conductance from spin blockade between the two dots as discussed in Section \ref{sec:dqd_model}.