\chapter{Spin Transport in Tunable Ferromagnet/Superconductor Junction}
\label{sec:SCFM}
\chaptermark{F-CNT-S}

%%%%%%%%  Chapter starts below %%%%%%%%%%%

In 1971, Tedrow and Meservey made their famous measurement of the spin polarization in a ferromagnetic nickel thin film \cite{Tedrow1971}. This was done by creating a tunnel junction between the thin nickel film and a thin aluminum film. The density of states of the aluminum film was used as a probe into the spin polarization of the nickel film by measuring the conductance through the junction. Tedrow and Meservey later used this same technique to characterize a number of ferromagnetic materials \cite{Tedrow1973}

%The motivation for the work in this chapter was to make the same type of measurement using a carbon nanotube quantum dot as a spin tunable tunnel barrier between a ferromagnetic and superconducting lead. In doing so, the spin resolved density of states of each material could be inferred from the differential conductance through the quantum dot as a function of magnetic field.

The goal in this work was to replace the insulating barrier in the Meservey measurement with a carbon nanotube quantum dot. In doing so, it becomes possible to use the levels on the dot to further tune spin transport through the junction. Differential conductance measurements through such a F-CNT-S can be made to measure the ferromagnet polarization, exchange coupling in the carbon nanotube, and the spin-resolved density of states of the superconductor. The work presented in this chapter represents the first attempt to measure transport in this type of carbon nanotube quantum dot device.

\section{Background}

The results of the Tedrow and Meservey experiment can be described with a simple theory based on the density of states of the materials \cite{Meservey1994}. Figure \ref{fig:MT_explanation}(a) shows schematically the density of states on each side of the tunnel barrier in a non-zero magnetic field applied parallel to the thin films separated by an insulator. The ferromagnet has a larger number of spins aligned with its magnetization (and the applied field). In the superconducting film, the density of states is split due to Zeeman splitting in the applied magnetic field. The energy to add a spin up (aligned with the field) particle is lower than adding a spin down particle. 

\begin{figure}
    \centering
    \includegraphics[width=0.6\textwidth]{scfmdots/MT_explanation.pdf}
    \caption{(a) Energy diagram showing the density of states of each material in the F-I-S junction. Figure adapted from \cite{Moodera2010}. (b) Superconducting density of states with Zeeman splitting. (c) Spin-dependent kernel used to calculate the tunneling current. (d) Differential conductance as measured through the junction. Different spin-resolved peaks are labeled $\sigma_{1-4}$. Figures (b)-(d) are adapted from \cite{Meservey1994}.}
    \label{fig:MT_explanation}
\end{figure}

To begin, the tunneling current through a normal-insulator-superconductor junction can be written as follows:

\begin{equation}
    \label{eq:nis_junction}
    I \sim N_n \int N_s(E)[f(E+eV) - f(E)]dE
\end{equation}

Where $N_n$ is the normal metal density of states which is indenpendent of energy in 2D and $f(E)$ is the Fermi function. $E$ is the Fermi level in the system and V is the applied bias voltage. The superconducting density of states $N_s$ can be written in the BCS form \cite{Tinkham1996}:

\begin{equation}
    \label{eq:bcs_dos}
    N_s(E) = \begin{cases}
    N_n(E)E\frac{E}{sqrt{E^2 \Delta^2}}, & |E| \geq \Delta.\\
    0, & |E| < \Delta.
  \end{cases}
\end{equation}

Taking the derivative of Equation \ref{eq:nis_junction} gives the differential conductance for the NIS junction:

\begin{equation}
    \label{eq:nis_diffcond}
    \frac{dI}{dV} \sim N_n \int N_s(E)[\frac{df(E+eV)}{dV}]dE
\end{equation}

In an applied magnetic field the spin up (parallel to applied field) and spin down (anti-parallel) contributions can be written separately.

\begin{equation}
    \label{eq:nis_diffcond_field}
    \frac{dI}{dV} \sim N_n \int N_s(E+\mu H)[\frac{df(E+eV)}{dV}]dE + N_n \int N_s(E-\mu H)[\frac{df(E+eV)}{dV}]dE
\end{equation}

Now that the expression for the differential conductance of an N-I-S junction is known, it is simple to infer the F-I-S results. Since the ferromagnet has a density of states that can be considered independent of the energy, like a normal metal in 2D, the only change is a factor to account for relative populations of spin up and spin down electrons.

\begin{equation}
    \label{eq:fis_diffcond_field}
    \frac{dI}{dV} \sim N_n \int N_s(E+\mu H)[a\frac{df(E+eV)}{dV}]dE + N_n \int N_s(E-\mu H)[(1-a)\frac{df(E+eV)}{dV}]dE
\end{equation}

Where $a$ is the fraction of spin up electrons in the ferromagnet. Figure \ref{fig:MT_explanation}(b) shows the superconducting density of states in non-zero field and (b) shows the spin dependent kernel in the integrals in Equation \ref{eq:fis_diffcond_field}. Finally, Figure \ref{fig:MT_explanation}(d) shows the resulting differential conductance through the F-I-S junction. 

The peak heights seen in Figure \ref{eq:fis_diffcond_field}(d), $\sigma_{1-4}$ can be used to calculate the polarization.

\begin{equation}
    \label{eq:polarization}
    P = 2a-1 = \frac{(\sigma_4 - \sigma_2) - (\sigma_1 - \sigma_3)}{(\sigma_4 - \sigma_2) + (\sigma_1 - \sigma_3)}
\end{equation}

Equation \ref{eq:polarization} is what Tedrow and Meservey derived to measure the spin polarization of a given ferromagnetic thin film using tunneling current through an F-I-S junction.

Now, it is important to consider what might be observed when replacing the tunnel junction with a carbon nanotube quantum dot. An energy diagram of the device can be seen in Figure \ref{fig:MT_proposed}.

\begin{figure}
    \centering
    \includegraphics[width=0.5\textwidth]{scfmdots/MT_proposed.pdf}
    \caption{Energy diagram showing the density of states and energy levels in a F-QD-S junction. Figure adapted from \cite{Moodera2010}.}
    \label{fig:MT_proposed}
\end{figure}

The levels on the dot should be non-degenerate due to exchange coupling with the ferromagnetic lead. The exchange coupling magnitude is not known in carbon nanotubes and could be determined using the proposed devices. Such a measurement would be useful in discussing spin transport through ferromagnetically contacted nanotubes such as in Section \ref{sec:spin_selection_field}. The magnitude of the conductance peaks sketched in Figure \ref{fig:MT_explanation}(d) will be modulated by bringing quantum dot levels with different spin polarizations in and out of resonance with the Fermi level on the magnetic and superconducting leads. Looking at how these conductance peaks evolve in the magnetic field at different gate voltages will give information about the exchange coupling between the nanotube and ferromagnet. A similar measurement has been made to confirm the ferromagnetic exchange coupling in \ce{InAs} quantum dots \cite{Hofstetter2010}

Measurements of the differential conductance as a function of $V_{bias}$ and applied field should show the conductance peaks sketched in Figure \ref{fig:MT_explanation}(d) move as a function of the applied field in the same was as observed by Tedrow and Meservey. If the spin texture of the resonant quantum dot level is known, this measurement will probe the density of states of the ferromagnet and superconductor materials.

\section{Characteristics of Samples}

Of the devices fabricated, two F-CNT-S devices yielded clean measurements at 4K, MT7 and SCFMH8. Basic properties of each device will be summarized here. These properties will be useful later in discussing the spin transport through each device.

\subsection*{MT7}

Sample MT7 was one of the earliest samples made for this project and motivated a lot of my later work on fabrication and experimental setup. It was made using dropcast nanotubes on a silicon substrate with sputtered cobalt (50nm) and niobium (60nm) leads. 

\begin{figure}
    \centering
    \includegraphics[width=0.5\textwidth]{scfmdots/mt7_sem.pdf}
    \caption{Scanning electron microscope image of sample MT7. The leads are alternating Co and Nb, starting with Co at the top. Each section of nanotube between to adjacent leads forms a 300nm long quantum dot. The nanotube is not visible because of the 30kV accelerating potential used to make the image.}
    \label{fig:mt7}
\end{figure}

The device is show in Figure \ref{fig:mt7}. Figure \ref{fig:mt7_gate} shows the conductance as a function of $V_{gate}$ measured at 4K. The inset of that figure shows some two-fold symmetry in the peaks. The spin degeneracy should be broken by the presence of the cobalt leads. The two-fold symmetry likely comes from orbital symmetry, which is assumed to not be broken by the magnetic contacts. 

\begin{figure}
    \centering
    \includegraphics[width=0.8\textwidth]{scfmdots/mt7_9-11_gateswp_4K.pdf}
    \caption{Conductance as a function of $V_{gate}$ in MT7. The inset shows a region with clear two-fold symmetry in the addition energy.}
    \label{fig:mt7_gate}
\end{figure}

Coulomb blockade features were measured in this device both above and below the critical field of the niobium contacts, which is just under 1T. Figure \ref{fig:mt7_coulomb} shows this data. Some important information about the nature of this quantum dot can be extracted from the data in Figure \ref{fig:mt7_coulomb}. In the normal state (B=1T), the energy level splitting is measured to be $\Delta E  = 2.0$meV. This corresponds to a calculated quantum dot size of 250nm, in good agreement with the image in Figure \ref{fig:mt7}. The measured addition energy is about 2.5meV, giving a charging energy of 0.7meV and a total capacitance of 100aF and lever arm of $\alpha_G = 3.8$. This puts MT7 in a wildly different coupling regime than the substrate grown nanotubes used in the rest of this thesis.

\begin{figure}
    \centering
    \includegraphics[width=0.8\textwidth]{scfmdots/mt7_9-11_vigate_4K.pdf}
    \caption{Conductance as a function of $V_{bias}$ and $V_{gate}$ in 0T (top) and 1T (bottom) magnetic field parallel to the leads.}
    \label{fig:mt7_coulomb}
\end{figure}

In the top plot of Figure \ref{fig:mt7_coulomb} the niobium lead is in the superconducting state. Due to the presence of the superconducting gap, the overall measured conductance is decreased and the coulomb diamonds no longer close at low bias. The splitting between diamond peaks gives a measure of the superconducting gap, $2\Delta = 3.5$meV. The expected gap can be calculated from BCS theory:

\begin{equation}
    \label{eq:bcs_gap}
    \Delta (T \rightarrow T_C) \approx 3.07 k_B T_C \sqrt{1 - T/T_C}
\end{equation}

Using this equation and the measured value of $T_C = 7$K, gives an expected gap of 1.2meV, which is in reasonably good agreement with the measured value, given the resolution of these measurements.

\subsection*{SCFMH8}

Sample SCFMH8 was one of the last samples made for this work. Nanotubes were grown directly on the substrate with predefined Mo markers and contacted with thermally evaporated permalloy (40nm) and sputtered Ti/Nb (5nm/60nm). An image of the device can be seen in Figure \ref{fig:scfmh8}.

\begin{figure}
    \centering
    \includegraphics[width=0.5\textwidth]{scfmdots/scfmg1_sem.pdf}
    \caption{Scanning electron microscope image of sample scfmh8. The three wide leads are Nb, while the three narrow leads are Py. These leads are connected to bonding pads made with electron beam lithography and thermally evaporated Cr/Au layers.}
    \label{fig:scfmh8}
\end{figure}

As fabricated, the thermally evaporated permalloy leads had contact resistances in the 100M$\Omega$ range. After fabrication, the samples were annealed at 325$^\circ$C for 3 hours in an Ar/\ce{H2} atmosphere. After annealing, both Py-CNT-Py and Nb-CNT-Nb junctions were around 100k$\Omega$. Tuning the back gate on this device was found to have no effect on the transport properties. It is possible that the high temperature growth or anneal damaged the gate oxide.

Figure \ref{fig:scfmh8_vi} shows the current and conductance as a function of the bias voltage. The sample is clearly in the Coulomb Blockade regime, but the level spacings are difficult to determine. A two probe resistance measurement was made across a niobium wire fabricated on the same chip as SCFMH8. The critical field in this wire was found to be 2T.

\begin{figure}
    \centering
    \includegraphics[width=0.8\textwidth]{scfmdots/scfmh8_vi_4K.pdf}
    \caption{Current (left) and conductance (right) as a function of $V_{bias}$ in sample SCFMH8.}
    \label{fig:scfmh8_vi}
\end{figure}

\section{Zeeman Splitting of Differential Conductance Peaks}

In calculating Equation \ref{eq:fis_diffcond_field}, a tunnel barrier with a height independent of energy was assumed. By replacing the tunnel barrier used in that model with a quantum dot, the assumption no longer holds. In order to capture the energy dependence of the quantum dot a term proportional to the density of states on the quantum dot must be added to the integral over the superconducting density of states. The data discussed in this section correspond to such a model.

Figure \ref{fig:mt7_bisweep_edit} shows the evolution of the differential conductance peaks in sample MT7 as a function of the magnetic field. Yellow dashed lines in the bottom plot serve as a guide to the eye for each of the differential conductance peaks.

\begin{figure}
    \centering
    \includegraphics[width=0.75\textwidth]{scfmdots/mt7_9-11_bisweep-edited.pdf}
    \caption{Top left: Coulomb diamonds in 0T. The white dashed line marks the gate voltage where the field sweep was taken. Top right: Coulomb diamonds in 2T parallel field. Bottom: Conductance as a function of $V_{bias}$ and parallel magnetic field. Here $V_{bias}$ was the fast sweep axis. Dashed lines represent approximate peak positions.}
    \label{fig:mt7_bisweep_edit}
\end{figure}

Due to low resolution of the data it is difficult to make any fits to the dashed yellow lines tracing the conductance peaks in Figure \ref{fig:mt7_bisweep_edit}. Similar data was taken for SCFMH8 and can be seen in Figure \ref{fig:scfmh8_bisweep_edit}

\begin{figure}
    \centering
    \includegraphics[width=0.75\textwidth]{scfmdots/scfmh8_15-16_bisweep-edited.pdf}
    \caption{Conductance as a function of $V_{bias}$ and parallel magnetic field in SCFMH8. $V_{bias}$ was the fast sweep axis. Dashed lines represent approximate peak positions. Magnetic field sweep direction is from left to right.}
    \label{fig:scfmh8_bisweep_edit}
\end{figure}

To more easily discuss the differential conductance as a function of field, vertical cuts across the data in Figures \ref{fig:mt7_bisweep_edit} and \ref{fig:scfmh8_bisweep_edit} are shown in Figure \ref{fig:bisweep_cuts_all}

\begin{figure}
    \centering
    \includegraphics[width=1.0\textwidth]{scfmdots/bisweep_cuts_both.pdf}
    \caption{Conductance as a function of $V_{bias}$ and parallel magnetic field in samples MT7 (left) and SCFMH8 (right). Conductance curves have been offset for clarity.}
    \label{fig:bisweep_cuts_all}
\end{figure}

Figure \ref{fig:bisweep_cuts_all} shows how the tunneling differential conductance evolves as a function of the magnetic field in each sample. Sample SCFMH8 shows very little movement of the peaks with the magnetic field. There are a number of reasons this might be the case. Most importantly, the splitting could be obscured by thermal noise in the system, which is on the order of 500$\mu$eV. Zeeman splitting of opposite spin levels on a nanotube quantum dot has been observed in the past to be of the same order of magnitude \cite{Tans1997}. This seem to be the most reasonable explanation. It is difficult to speculate any more on what splittings are expected from the Coulomb blockade features alone without the device being gate tunable. Looking closely at Figure \ref{fig:scfmh8_bisweep_edit} it is possible that there is some evidence of the superconducting gap closing in the outermost conductance peaks near $B=2T$.

In Figures \ref{fig:bisweep_cuts_all} and \ref{fig:mt7_bisweep_edit} it is clear that there is some field dependent movement of the conductance peaks in sample MT7. Again, it is important to remember a change in conductance at low bias is not solely due to the superconducting gap, but rather a combination of the gap and Coulomb blockade. From $B=0$T to 0.5T conductance peak 1 (labelled in Figure \ref{fig:mt7_bisweep_edit}) shifts from $V_{bias}=-4.2meV$ to $V_{bias}=3meV$. This agrees well with the closing of the calculated superconducting gap of 1.2meV. Similar features are observed in the other three conductance peaks. At fields above 0.5T, peaks 1-3 have a slight field dependance. Peaks 1 and 2 show positive slopes of 1.5 and 0.3 meV/T, respectively. Peak 3 has a slope of -0.1meV/T. Qualitatively this makes sense in terms of the Coulomb blockade. The magnitudes of the slopes are consistent with Zeeman splitting energies, $g \mu_B B$, in carbon nanotubes with a g-factor of 2. This also suggests that the levels are spin split due to exchange coupling with the ferromagnet, giving a hint at the magnitude of the coupling.

Further measurements to resolve these features at lower temperatures will be necessary for a more quantitative analysis.

\section{Hysteretic Switching of Quantum Dot Conductance in an External Field}
\label{sec:scfm_switching}

After not observing any clear field dependence in the differential conductance of sample SCFMH8, additional measurements were performed. For these measurements the fast and slow scan axes were switched. Sweeping the magnetic field, at a fixed bias, while measuring current through the quantum dot revealed some interesting behavior. The full data set is seen in Figure \ref{fig:scfmh8_positive_field_switching}.

\begin{figure}
    \centering
    \includegraphics[height=0.8\textheight]{scfmdots/scfmh8_15-16_fieldswp_p_4K.pdf}
    \caption{Current as a function of field measured at $V_{bias}$ between -2.5 and 2.5mV. The same measurement was made from 0 to -3T with similar results. See Figure \ref{fig:scfmh8_peak_positions} for a summary of the switching behaviors at both field polarities.}
    \label{fig:scfmh8_positive_field_switching}
\end{figure}

There are three features to note in Figure \ref{fig:scfmh8_positive_field_switching}. There are two changes in conductance, at 0.5T and 2.0T, apparent in each sweep. Based on the measured niobium critical field, it is obvious to associate the 2T change in conductance with the niobium leads transitioning to the normal state. The switch at 0.5T is 10-50 times larger than the expected coercive field for permalloy leads of this approximate size and shape \cite{Aurich2010, Preusche2009}. Additionally, both of these switching events show a hysteresis with the magnetic field sweep direction. 

The curves in Figure \ref{fig:scfmh8_positive_field_switching} can be compared to horizontal cuts across the data in Figure \ref{fig:scfmh8_bisweep_edit} to reveal the difference in the two measurements more clearly. A comparison of the two data sets is seen in Figure \ref{fig:scfmh8_fast_slow_compare}. 

\begin{figure}
    \centering
    \includegraphics[width=1.0\textwidth]{scfmdots/scfmh8_fast_slow_compare.pdf}
    \caption{A comparison of the current versus field data with fast and slow sweep axes reversed.}
    \label{fig:scfmh8_fast_slow_compare}
\end{figure}

As a control, the same measurement seen in Figure \ref{fig:scfmh8_positive_field_switching},  was made on a Nb-CNT-Nb quantum dot on the same nanotube. A subset of those results are compared with the Py-CNT-Nb quantum dot in Figure \ref{fig:scfmh8_superconductor_compare}.

\begin{figure}
    \centering
    \includegraphics[width=1.0\textwidth]{scfmdots/scfmh8_superconductor_compare.pdf}
    \caption{Left column shows current as a function of magnetic field for the Py-CNT-Nb quantum dot. Right column shows the same data for a Nb-CNT-Nb quantum dot on the same nanotube. Note the appearance of a conductance change at 0.5T in some of the Nb-CNT-Nb data with no hysteresis.}
    \label{fig:scfmh8_superconductor_compare}
\end{figure}

Figure \ref{fig:scfmh8_superconductor_compare} shows clearly that the hysteresis is missing from the field sweep in the Nb-CNT-Nb quantum dot. The conductance change associated with the superconducting critical field is still seen at 2T in each measurement. The sweep at $V_{bias}=-1$mV data shows the switching behavior at 0.5T remains in this quantum dot. The feature is not nearly as well-defined and the hysteresis is completely absent in this quantum dot. The full Nb-CNT-Nb dataset is plotted in Figure \ref{fig:scfmh8_all_superconducting}. 

\begin{figure}
    \centering
    \includegraphics[height=0.8\textheight]{scfmdots/scfmh8_2-1_fieldswp_s_4K.pdf}
    \caption{Current as a function of field measured at $V_{bias}$ between -2.5 and 2.5mV for a Nb-CNT-Nb quantum dot. }
    \label{fig:scfmh8_all_superconducting}
\end{figure}

\subsection{Discussion}

A schematic of the full device with both measured quantum dots highlighted is shown in Figure \ref{fig:scfmh8_illustration}(a). 

\begin{figure}
    \centering
    \includegraphics[width=0.8\textwidth]{scfmdots/scfmh8_illustrated.pdf}
    \caption{(a) A schematic of sample SCFMH8. Purple leads are Py, blue are Nb. The two quantum dots measured for Figure \ref{fig:scfmh8_positive_field_switching} and \ref{fig:scfmh8_all_superconducting} are circled in red. (b) An energy level diagram representing the tunneling at $V_{bias}=0$.}
    \label{fig:scfmh8_illustration}
\end{figure}

It is difficult to determine the source of the hysteresis in these current versus field measurements. To begin, several possibilities can be ruled out. First, the effect cannot be due to the switching of the ferromagnetic lead magnetization. As mentioned above, the coercive field of Py is expected to be much smaller than 0.5T. Second, the field is swept repeatedly from 0 to 3T and back. There should be no switching of the magnetization unless the polarization of the field is changed.

We will rule out subgap transport phenomena, such as Kondo states \cite{Nygard2000, Grove-Rasmussen2007, Jarillo-Herrero2005}, cotunneling \cite{Liang2002}, and Andreev reflection \cite{Pillet2010} due to the magnitude of the field and relatively high temperature. The change in conductance, at least at 0.5T, could be attributed to a level coming into resonance with the edge of the superconducting gap due to Zeeman splitting. A diagram of the process is seen in Figure \ref{fig:zeeman_split_resonance}.

\begin{figure}
    \centering
    \includegraphics[width=0.8\textwidth]{scfmdots/zeeman_split_resonance.pdf}
    \caption{(a) An energy level diagram in zero field. (b) The same diagram with Zeeman splitting of the quantum dot leves and the superconducting density of states.}
    \label{fig:zeeman_split_resonance}
\end{figure}

There are a few reasons to rule out the situation in Figure \ref{fig:zeeman_split_resonance}. First, the conductance switching exists independent of the bias voltage. Second, there is no reason to expect the process depicted in Figure \ref{fig:zeeman_split_resonance} to be hysteretic. Finally, it is unlikely for the Nb-CNT-Nb to have the same configuration of levels.

To investigate the bias dependence of the switching behavior, the field values at which the conductance switching occurs are plotted in Figure \ref{fig:scfmh8_peak_positions}. Based on the data, it does not appear that the switching behavior depends on the bias. Both the switching positions and the width of the hysteresis appears to be constant as a function of $V_{bias}$. It is also interesting to note that the Nb-CNT-Nb switching fields correspond to those measured in the Py-CNT-Nb down sweep.

\begin{figure}
    \centering
    \includegraphics[width=0.8\textwidth]{scfmdots/scfmh8_peak_positions.pdf}
    \caption{Position of the hysteretic switching events as a function of $V_{bias}$ at positive and negative magnetic field values. Note, the field was not swept through 0. Positive field data was taken for all bias voltages, then negative field data. This is done to separate the feature at 0.5T from any effects related to the switching magnetization of the leads, as discussed in Section \ref{sec:TMR}}
    \label{fig:scfmh8_peak_positions}
\end{figure}

One final possibility is that the hysteresis is not due to any transport phenomena, but rather a thermodynamic effect due to the arrangement of leads on the substrate. Looking at the superconducting quantum dot that was used as a control, it is separated from the large ferromagnetic leads by nearly 1$\mu$m. Conversely, the superconducting lead used in the F-CNT-S junctions is surrounded on both sides by larger ferromagnetic leads. 

In light of this discussion, there is no clear physical picture for either the switching at 0.5T or the hysteresis observed at 0.5T and 2.0T. Future measurements looking at the behavior of this feature as a function of $V_{gate}$ at lower temperatures should determine if the switching is a result Zeeman splitting of levels on the dot. If it is, the physical origin of the hysteresis may be very interesting. Additionally, different device geometries (ordering the ferromagnetic and superconducting leads differently) could eliminate the possibility of a thermodynamic switching in the superconducting lead, rather than a transport phenomena.

This chapter represents the first attempt at measuring spin-dependent transport properties through a F-CNT-S junction. The Zeeman splitting of conductance peaks and closing of the superconducting gap demonstrates that both superconductivity and spin non-degenerate quantum dot levels exist in the device. The measurement of hysteretic conductance splitting suggests the existence of new transport phenomena in this device that provides the basis for future work.