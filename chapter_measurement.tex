\chapter{Low Temperature Transport Measurements}
\label{sec:measurement}
\chaptermark{Measurements}

In this thesis work a variety of low temperature and low noise measurement techniques have been used. This chapter will summarize these techniques with a focus on my own work and making the measurements reproducible by future lab members. 

\section{Transport Measurements}

All of the devices measurements carried out for this work have been two-probe, voltage biased measurements. The majority of measurements were DC, based on the fact that problems like gate leakage current and sources of electronic noise are much more easily identified in a DC measurement. A few AC measurements were made, especially in the beginning of this work. Each measurement setup will be reviewed here.

\subsection{DC Current Measurements}
\label{sec:DC}

Figure \ref{fig:dc_measurement} shows a typical DC measurement setup. The sample is loaded into a cryostat and accessed through a 24 channel BNC breakout box. The bias voltage is typically supplied by a National Instruments Digital Acquisition (DAQ) board. In some cases, old Kepco programmable voltage supplies were used. The Kepco supplies were found to be much lower noise, due to the opto-couplers used, but the DAQ board were preferred for their speed and ease of programming. Both the Kepco and DAQ boards have output ranges of $\pm$10V. To utilize the full dynamic range of the voltage sources, a 100:1 voltage divider is used at the output of the bias voltage source. Additionally, a 1kHz, two-pole, RC low pass filter is used to prevent voltage spikes that occur while ramping the output voltage from reaching the sample.

\begin{figure}
    \centering
    \includegraphics[width=0.6\textwidth]{measurements/dc_measurement.png}
    \caption{Diagram of typical DC measurement setup.}
    \label{fig:dc_measurement}
\end{figure}

Current through the nanotube is measured by a current to voltage converter with an output of 1V/$10^7$A. For this work two different amplifiers have been used, an Ithaco 1211, and a battery-powered homemade amplifier, these are discussed further in Section \ref{sec:CV_amp}. The output of the current to voltage amplifier is measured either by the NI DAQ board or a Keithley 2182A nanovoltmeter. Most measurements were made using the NI DAQ board for its speed and ability to save time series data for every measurement. To eliminate aliasing effects in the DAQ measurements a 10kHz RC filter was used for anti-aliasing. Measurements were taken at 21.6kHz. To sample the current, data was collected at 21.6kHz for up to one second. The time series data was dumped to a binary file and the current value was calculated from the mean of the acquired data. Saving all the time series data allows for further analysis, such as 1/f and shot noise, to be done later. For differential conductance measurements, the DC current data was numerically differentiated. 

\subsection{AC Conductance Measurements}

AC measurements take advantage of lockin amplifiers to eliminate noise and make direct measurements of differential conductance. Two lockin amplifiers were used for this work, an analog Princeton Applied Research 124A and a digital SRS830. The SRS830 is a standard choice because of its easy setup and remote programming. The 124A, however, has much less output noise and was used for most of this work. The output of the 124A was read by an NI DAQ board in the same way as described in Section \ref{sec:DC}.

\begin{figure}
    \centering
    \includegraphics[width=0.6\textwidth]{measurements/ac_measurement_1.png}
    \caption{Diagram of typical DC measurement setup.}
    \label{fig:ac_measurement}
\end{figure}

Samples are biased in the same way as the DC measurement with addition of an AC signal taken from the output of the lock in amplifier. The AD and DC bias voltages are added together using a passive AC-DC adder, described in Section \ref{sec:adder}. 

Because the lock in amplifier can only be used to measure voltage signals, the current to voltage amplifier is still needed. The output of the current to voltage amplifier is sent to the input of the lockin. Using this setup, the differential conductance can be directly measured.

Some care must be taken to ensure the lockin signal is not being attenuated by any of the filtering in the AC-DC adder, cryostat, or current to voltage amplifier.  In this work, lockin signals of 27Hz were used, which was well below the cutoff frequency of any filtering in the experiment.

\subsection{Custom Electronics}

\subsubsection*{Current Amplification}
\label{sec:CV_amp}

The primary sources of noise in making low current DC measurements come from the 60Hz AC power lines and operational amplifiers in the current to voltage amplifier. The first source of noise can be largely eliminated by using an amplifier with battery power. While it is not possible to completely eliminate op amp noise, it can be reduced by choosing designing circuits using low noise amplifiers, metal film resistors, and appropriate filtering. 

A custom made current to voltage amplifier was designed for our lab by Chris Merchant. It uses two OPA27GP op amps in series before and after the sample. The design is two simple transimpedance amplifiers in series with the sample. For this work, the design was expanded upon by adding power line filtering, a low pass filter on each amplifier, and a simplified output configuration. The design can be seen in Figure \ref{fig:current_amp}. With this amplifier, the sample is symmetrically biased by having the same amplifier impedance on each side.

\begin{figure}
    \centering
    \includegraphics[width=0.75\textwidth]{measurements/amplifier.pdf}
    \caption{Circuit diagram and photo of the current to voltage amplifier. The amplifier has a fixed gain of 1V/$10^7$A. Each power line is filtered with a 0.47$\mu$F capacitor. The 5pF capacitor in the feedback loop forms a low pass filter with the 10M$\Omega$ resistor, with a cutoff frequency of about 3kHz.}
    \label{fig:current_amp}
\end{figure}

The output of each amplifier (one measuring I and the other -I) goes to a 9-pin Dsub connector along with the $\pm$12V power supply lines with the remaining pins grounded. From this cable the signal is sent to the NI DAQ for measurement. This configuration provides some interesting flexibility. In the simplest case, each output is sent to an antialiasing filter and measured separately. Measuring each amplifier separately allows noise measurements to be made by looking at the cross correlation of the two signals \cite{Merchant2009}. By measuring the current at with two separate amplifier circuits, the amplifier and environmental noise from each is cancelled, leaving only noise from the sample.

When not making noise measurements, it is simpler to measure just one signal for the current. To do this, while still taking advantage of the noise canceling, an active subtractor circuit was constructed. The subtractor circuit can be seen in Figure \ref{fig:subtractor}. 

\begin{figure}
    \centering
    \includegraphics[width=0.6\textwidth]{measurements/active_adder.pdf}
    \caption{Circuit diagram and photo of the active subtractor/adder circuit. The DPDT switch is used to change between adder/subtractor circuits. The output state also features an antialiasing filter with 10kHz cutoff not drawn here.}
    \label{fig:subtractor}
\end{figure}

Both the amplifier and subtractor are run using the same $\pm$12V battery power supply, which is housed in a separate aluminum enclosure and connected with a 5-pin cable. Two identical battery sources were constructed so that one could charge while the other is in use. Measurements can be paused in the software control during battery changes. Most data in this thesis was taken using this setup.

%A major drawback of this current amplifier/subtractor setup is that it fails for samples with a resistance above 20M$\Omega$. This makes this amplifier difficult to use for tunnel probe measurements, as tunnel probe junctions typically have resistances of the same order of magnitude. For tunnel probe measurements, and most lockin measurements, the Ithaco 1211 amplifier was used. The reason for this is that there is a 20M$\Omega$ short across the sample through the input of the subtractor circuit. This should be fixed for future measurements by the addition of current blocking diodes in the subtractor circuit. In carbon nanotube quantum dot measurements this was not a problem, because the resistances of interest are not typically so large. 

\subsubsection*{AC-DC Adder}
\label{sec:adder}

When making AC measurements that incorporate a DC bias voltage, some care must be taken to add the two voltage biases together in the correct way. Several circuits were tested to do this, from a simple BNC tee, to an active adder circuit like the one seen in Figure \ref{fig:subtractor}. The design that yielded the least noise, most accurate output, and shortest decay times is seen in Figure \ref{fig:ac_dc}.

\begin{figure}
    \centering
    \includegraphics[width=0.6\textwidth]{measurements/ac_dc.pdf}
    \caption{Circuit diagram and photo of the AC-DC adder. The voltage is added in the first stage, while the second state consists of a two-pole RC low pass filter with a cutoff frequency of 1kHz to suppress any switching noise from the voltage sources.}
    \label{fig:ac_dc}
\end{figure}

The 470$\mu$F capacitor on the AC input prevents any DC bias from the lockin amplifier from affecting the measurements. The second stage is a two voltage dividers (100:1 AC and 100000:1) DC which also serves as the adder circuit by adding the two currents through the 100$\Omega$ resistor to ground. The last stage is a two-pole low pass filter. This circuit works well with AC signals up to 100Hz, but is best used with signals of 10-20Hz.

\section{Cryogenics}

A number of cryostats and superconducting magnets were used in this work. This section will briefly review the various components used.

\subsection{Cryostat Wiring}

\begin{figure}
    \centering
    \includegraphics[width=0.9\textwidth]{measurements/sample_wiring.png}
    \caption{(a) Parallel magnetic field sample holder. (b) 24-pin BNC breakout box for Fischer connectors.}
    \label{fig:wiring}
\end{figure}

All the cryostats discussed below are wired in the same way. Samples are mounted on 28-pin chip carriers, which are placed in custom built sample holders with custom circuit board designed by our lab. The sample holders, seen in Figure \ref{fig:wiring}(a), can are screwed onto the cryostat at the coldest point. Each sample holder has a mount for a chip carrier, as well as a RuOx or Pt temperature sensor and heater. Lines from the sample are connected using 25 pin Dsub connectors mounted on the sample holder and at the coldest point on the cryostat. From the Dsub connectors the signal moves through twisted pair ribbon cables to a Fisher connector mounted on the cryostat at room temperature. A 24-lead cable connects the cryostat to a BNC breakout box, seen in Figure \ref{fig:wiring}(b). None of the cryostats discussed here have low temperature filtering other than that formed by the impedance of the ribbon cable. Typical cutoff frequencies for this setup are around 10kHz, which dictates the choice of measurement frequency in DC measurements. Adding additional high frequency filtering would certainly improve the noise in our measurements by lowering the electron temperature in the samples and should be implemented in future work.

\subsection{Dunker}

\begin{figure}
    \centering
    \includegraphics[width=0.9\textwidth]{measurements/dunker.png}
    \caption{Liquid helium dunker with parallel field sample holder (far right).}
    \label{fig:dunker}
\end{figure}

A simple 4He dunker was built for our lab several years ago by Chris Merchant. The dunker, seen in Figure \ref{fig:dunker} consists mainly of a vacuum can, sample holder, and ribbon cable connected to a 24-pin Fischer connector at room temperature. The dunker is very useful for quick measurements, and is extremely reliable for reaching 4K. It is small enough to fit in just about any magnet or transfer dewar and can be loaded and cooled down in about an hour. This dunker was used for many of the measurements in this thesis.

\subsection{3He Cryostats}

The primary cooling in 3He cryostat is from evaporative cooling of 3He liquid by cryopumping with a charcoal sorb. Before this is possible the 3He must be liquified. This is done by first pumping on a small volume of 4He known as the 1K pot. This pot will typically reach temperatures between 1.2 and 1.8K. The 3He gas, stored in a charcoal sorb in a separate volume, is released from the charcoal by heating the sorb to about 30K. The 3He gas then passes through the 1K pot, which is cold enough to condense the 3He gas into a liquid. The 3He liquid collects in a 3He pot, to which the sample has been thermally anchored. Once all the gas is released from the sorb, the sorb is cooled to 4K. The cold sorb the acts as a cryopump on the liquid 3He. This cryopumping cools the 3He pot to its base temperature of 250mK \cite{Balshaw2001}.

\subsubsection*{Oxford Heliox}

\begin{figure}
    \centering
    \includegraphics[width=0.9\textwidth]{measurements/oxford_he3.png}
    \caption{Oxford Heliox with the sample holder removed from helium 3 pot (far left).}
    \label{fig:heliox}
\end{figure}

The Oxford 3He cryostat was used in this work for some very early measurements. After that it developed a leak in the 1K pot pumping line that made it very inconvenient for this work. The typical hold times were about 24 hours, 12 hours on a bad day.

The operation of these Oxford cyrostats is well-documented elsewhere. Temperature control is mainly done with an ITC503 temperature controller which monitors the temperature sensors and adjusts needle valve on the 1K pot line to keep the pot at the proper temperature during operation.

\subsubsection*{RMC 3He Cryostat}

\begin{figure}
    \centering
    \includegraphics[width=0.9\textwidth]{measurements/rmc_he3.png}
    \caption{Photo of the RMC 3He cryostat with the sample holder removed from the 3He pot.}
    \label{fig:rmc}
\end{figure}

After both of our Oxford cryostats were found to have leaks, we obtained this cryostat on loan from a colleague. This cryostat had not been used for a decade before our lab adopted it. In order to make it compatible with our sample holders and breakout boxes, we rewired the cryostat with a 24-pin Fischer connector at room temperature and a 25-pin miniature D-sub connector anchored to the 3He pot. The original temperature control wiring was left in place, connected to a separate 19-pin connector which was then connected to a custom breakout box. Heater power was supplied with $\pm$12V power supplies on the breakout box. The RuOx and Pt 3He pot temperature sensors were read with a Neocera LTC21 temperature controller.

Instructions for cooling down the RMC 3He cryostat can be found in Appendix \ref{sec:rmc}

\subsection{Dilution Refrigerator}

Dilution refrigerators rely on a mixture of 3He and 4He to achieve temperatures as low as 5mK. Our Kelvinox system is said to be capable of 7mK base temperatures, although, due to various problems with the IVC vacuum I have been unable to cool it below 70mK.

The process by which the dilution fridge achieves its based temperature is easiest to understand by following the path of the gas mixture through the system. From room temperature the mixture is cooled to 4K traveling in the condenser line through the LHe bath in the magnet dewar. The mixture then passes through the 1K pot where 3He is condensed. The condensed mixture then passes through the primary impedance, designed to control the flow rate to the mixing chamber. After the primary impedance are a series of heat exchangers which use outgoing cold gas to cool the incoming mixture to 50mK. From there the mixture moves into the mixing chamber. The mixing chamber contains two phases of the 3He/4He mixture, a dilute phase and a concentrated phase. The most important cooling stage of the dilution fridge is the evaporating of 3He from the concentrated phase into the dilute phase in the mixing chamber. 3He the moves from the dilute phase in the mixing chamber into the still, which is held at about 600mK. Osmotic pressure from the lower 3He concentration in the still drives this movement. Finally the mixture is pumped through the heat exchangers and out of the fridge through the still pumping line. It then passes through cold traps to remove any impurities before reentering the condenser line \cite{Balshaw2001}.

\subsubsection*{Oxford Kelvinox}

\begin{figure}
    \centering
    \includegraphics[width=0.9\textwidth]{measurements/dilution_fridge.png}
    \caption{Oxford Kelvinox dilution refrigerator with custom built three sample holder (far right).}
    \label{fig:kelvinox}
\end{figure}

The Oxford Kelvinox used in our lab is pictures in Figure \ref{fig:kelvinox}. This system was not used for about 8 years before I cooled it down in 2014. It required several repairs to the temperature control wiring and pump power supplies, but the gas handling systems were found to be in very good condition. After several cool downs, it became apparent that the system had a leak into the IVC from the top plate. 

Instructions on cooling down the Kelvinox dilution refrigerator can be found in Appendix \ref{sec:kelvinox}.

\subsection{Superconducting Magnets}

\subsubsection*{Oxford Magnet}

This magnet was purchased along with the Oxford Heliox and Kelvinox cryostats. Those two cryostats are designed to fit easily into the magnet bore with the samples sitting at the center of the field as calibrated by the manufacturer. It is an 8T superconducting magnet mounted in an Oxford liquid helium dewar with a liquid nitrogen jacket. Current to the magnet is supplied by a matching Oxford IPS120 magnet power supply. This magnet was used for all of the measurements in Chapter \ref{sec:FMCNTQD}.

\subsubsection*{BTI Magnet}

This magnet was taken from a discarded BTI SQUID system. Similar to the Oxford magnet, it is a 9T superconducting magnet mounted in a liquid helium dewar with a liquid nitrogen jacket. This setup was assembled by, and borrowed from, Dan Reich's lab along with the RMC cryostat. Since the magnet could not be used with its original controller, a custom controller was assembled using an AMI controller with a HP power supply and custom made energy absorber/polarity switch. Upon receiving this system, it was tested and wired to a new BNC interface box to be controlled by a NI DAQ board along with transport measurements described above. This magnet was used for the data presented in Section \ref{sec:scfm_switching}.