\chapter{Introduction}
\label{sec:intro}
\chaptermark{Introduction}

This thesis work focuses on a narrow range of fabrication techniques and electronic transport measurements on a specific type of mesoscopic device, carbon nanotube quantum dots. In this introduction, the motivations behind this work will be explained along with an outline of what is to be discussed. 

\section{Transport Spectroscopy in Quantum Dots}

Using transport spectroscopy to probe low energy density of states can yield a great deal of insight into the nature of the materials being probed. A simple conductance measurement through two materials and a tunnel barrier measures a convolution of the density of states for each of the materials, $N_1(E)$ and $N_2(E)$.

\begin{equation}
    \label{eq:conductance}
    G \sim \int N_1(E) N_2(E)\frac{df(E+eV)}{dV}dE
\end{equation}

At low temperatures, the derivative of the Fermi function is approximated by a delta function and provides a sharp kernel for this convolution. By measuring the conductance across the junction, much can be learned about the nature of the materials, such as in the work of Tedrow and Meservey on measuring the polarization of magnetic materials \cite{Tedrow1971}

By using making a two junction device, and introducing a constriction in the intermediate material, either through local gating or by using low dimensional materials, an interesting situation is created. At low temperatures, the confinement of electrons in the constriction will dominate the transport. Consider an electron confined within a one dimensional length, $L$. The electron has energies on the order of $E \sim \hbar^2k \pi^2/2mL^2$. For length scales smaller than 100nm and $k_B T < E$ ,the quantum mechanical levels in the constriction dominate the electron transport through the device. The measured conductance will show discrete levels corresponding to the filling of the quantum levels in the constriction. This device is called a quantum dot.

Calculating the electron transport properties through a quantum dot is considerably more complicated than the application of Equation \ref{eq:conductance}. At small sizes and low temperatures, not only is the filling of the quantum levels in the dot important, but the electron-electron interactions on the dot must be considered, as well as the interactions between the dot, the two materials connected to it through tunnel barriers, and any electrostatic gates. The Hamiltonian for such a quantum dot with $N$ electrons looks like \cite{Ihn2004}:

\begin{equation}
\label{eq:full_hamiltonian}
    H_N = \sum_{n=1}^N \left( \frac{\mathbf{p}_n^2}{2m^*} - e \int_V dV \rho_{ion}(\mathbf{r})G(\mathbf{r}_n, \mathbf{r}) + \frac{e^2}{2} G(\mathbf{r}_n, \mathbf{r}) - e \sum_i \phi_i \alpha_i (\mathbf{r}_n) + e^2 \sum_{m=1}^{n-1} G(\mathbf{r}_m, \mathbf{r}_n) \right)
\end{equation}

In order, these terms describe the kinetic energy of the electrons, the energy of interactions with fixed ions in the system, image charge potential for an individual electron, changes in potential energy from interaction with $i$ gate electrodes, and a self-energy term that should be renormalized away. Obviously, comparing every experiment to this model is impractical and unnecessary. Section \ref{sec:constant_interaction_model} will introduce the simplest model for capturing quantum dot behavior, called the constant interaction model. In that, the Hamiltonian above is simplified such conductance through the dot only depends on the energy levels on the dot, as calculated with basic quantum mechanics, and the total capacitance of the device, which takes the place of all the electrostatic terms in Equation \ref{eq:full_hamiltonian}. Chapter \ref{sec:CNT} will discuss the basic behavior, at room temperature and low temperatures, of carbon nanotube quantum dot devices in terms of the constant interaction model.

The constant interaction model can be expanded by adding asymmetric, spin dependent tunnel barriers, and spin interactions between electrons in different energy levels confined to the same dot. This is the case when quantum dots are formed using ferromagnetic and superconducting materials. The interplay between magnetism, superconductivity, and transport through a quantum dot is the focus of much of this thesis. 

\section{Working with Carbon Nanotubes}

Carbon nanotube quantum dots are an ideal platform for making the type of transport measurements described above. They are inherently one dimensional conductors with diameters on the order of 1nm. The large aspect ratio of carbon nanotubes (>1000:1) leaves plenty of room to fabricate the metallic contacts needed for transport measurements along the nanotube length.

Building carbon nanotube quantum dot devices tends to be a very personal experience. Each young scientist has a list of recipes, opinions, and techniques with a wide range of scientific and quasi-religious motivations. Occasionally, this effort results in gorgeous datasets and rich physics. Once this happens, the hundreds of devices built, and subtle techniques tested, leading to the publication are quickly forgotten in favor of the results. The lack of transparency hinders reproducibility and progress in the field. An effort has been made in this thesis to explain, in detail, all of the techniques used in fabrication and comment on their reproducibility.

\section{Current Work}

Chapters \ref{sec:growth} and \ref{chap:contacts} document a variety of nanotube growth and contact fabrication techniques tested in this work. Chapter \ref{sec:growth} discusses  efforts made in our lab to replicate a number of nanotube growth recipes. In Chapter \ref{chap:contacts}, statistics on the devices made, their fabrication steps, and resulting measurements are discussed. The large volume of devices fabricated allows for a meaningful comparison of fabrication techniques that has not previously been seen. Improved imaging and sample processing techniques are also developed in this chapter. Additionally, a discussion of noise sources in the resulting devices is presented. Recommendations for fabrication are made based on the data analysis.

Chapters \label{sec:FMCNTQD} and \label{sec:SCFM} discuss spin-dependent, low-temperature, transport measurements on carbon nanotube quantum dot devices with ferromagnetic and superconducting contacts. In ferromagnetic devices, we were able to test a number models proposed in previous work and observe the appearance of conductance suppression and negative differential conductance based on spin selection rules. The F-CNT-S show evidence of proximity induced superconductivity as well as magnetic field dependent fluctuations in transport through the quantum dot. These measurements are the first attempt to analyze conductance through such a device.

