\chapter{Introduction}
\label{sec:intro}
\chaptermark{Introduction}

A quantum dot is a metal or semiconductor structure in which electron levels are quantized due to confinement in all three dimensions. Confinement of electrons in a potential well is one of the first problems taught in introductory quantum mechanics. Like most introductory level problems, at first glance it appears to be an unrealistically simple model. Contrary to that first impression, potential wells can be simple to fabricate and show a remarkable array of physical phenomena. In the simplest picture, electrons confined to the potential well can be thought of as an artificial atom, with each additional electron added filling the lowest unoccupied energy level and subject to spin and orbital degeneracies. 

Fabricating a potential well requires confining electrons within a physical constriction  with tunnel barriers. This is clear from the results of the infinite well problem, where the energy levels are proportional to $1/L^2$. As long as the electron temperatures are lower than the energy level spacing, the quantum level spacing can be observed by measuring electrons hopping into and out of the constriction. The important question in fabrication becomes, what sizes are small enough? The simplest cold temperature to reach in a modern laboratory is 4K, the temperature of a liquid helium bath. $k_B T$ provides a sensible energy scale from which to derive typical quantum dot sizes. 

Three dimensional nanoparticles must be grown or patterned down to dimensions of approximately 10nm and become very difficult contact electrically \cite{Ralph1997, Davidovic1998}. In two dimensions, defining a quantum dot requires complex gate structures on top of two dimensional conductors constricting electrons to areas with dimensions less than 100nm \cite{McEuen1992, Klein1995}. Carbon nanotubes, however, are intrinsically one dimensional conductors with diameters of approximately one nanometer. In this case, only the length along the tube axis must be constricted to a few hundred nanometers to observe quantum levels at 4K. Making this constriction can be as simple as fabricating two metal contacts to the nanotube, where the metal/nanotube interface provides the confinement potential. The same metal contacts that define the quantum dot make it possible to perform electronic transport measurements through the quantum dot. In this way the quantum mechanical structure is revealed by observing the tunneling of electrons through the dot. All of this, combined with simple carbon nanotube growth techniques, make nanotubes an ideal platform for studying quantum dot physics.

This thesis work focuses on fabrication techniques and electronic transport measurements on this specific type of mesoscopic device, carbon nanotube quantum dots. In this introduction, the motivations behind this work will be explained along with an outline of what is to be discussed. 

\section{Transport Spectroscopy in Quantum Dots}

Using transport spectroscopy to probe low energy density of states can yield a great deal of insight into the nature of the materials being probed. A simple conductance measurement through two materials and a tunnel barrier measures a convolution of the density of states for each of the materials, $N_1(E)$ and $N_2(E)$.

\begin{equation}
    \label{eq:conductance}
    G \sim \int N_1(E) N_2(E)\frac{df(E+eV)}{dV}dE
\end{equation}

At low temperatures, the derivative of the Fermi function is approximated by a delta function and provides a sharp kernel for this convolution. By measuring the conductance across the junction, much can be learned about the nature of the materials, such as in the work of Tedrow and Meservey on measuring the polarization of magnetic materials \cite{Tedrow1971}

By using making a two junction device, and introducing a constriction in the intermediate material, either through local gating or by using low dimensional materials, an interesting situation is created. At low temperatures, the confinement of electrons in the constriction will dominate the transport. Consider an electron confined within a one dimensional length, $L$. The electron has energies on the order of $E \sim \hbar^2k \pi^2/2mL^2$. For length scales smaller than 100nm and $k_B T < E$ ,the quantum mechanical levels in the constriction dominate the electron transport through the device. The measured conductance will show discrete levels corresponding to the filling of the quantum levels in the constriction.

Calculating the electron transport properties through a quantum dot is considerably more complicated than the application of Equation \ref{eq:conductance}. At small sizes and low temperatures, not only is the filling of the quantum levels in the dot important, but the electron-electron interactions on the dot must be considered, as well as the interactions between the dot, the two materials connected to it through tunnel barriers, and any electrostatic gates. The Hamiltonian for such a quantum dot with $N$ electrons looks like \cite{Ihn2004}:

\begin{equation}
\label{eq:full_hamiltonian}
    H_N = \sum_{n=1}^N \left( \frac{\mathbf{p}_n^2}{2m^*} - e \int_V dV \rho_{ion}(\mathbf{r})G(\mathbf{r}_n, \mathbf{r}) + \frac{e^2}{2} G(\mathbf{r}_n, \mathbf{r}) - e \sum_i \phi_i \alpha_i (\mathbf{r}_n) + e^2 \sum_{m=1}^{n-1} G(\mathbf{r}_m, \mathbf{r}_n) \right)
\end{equation}

In order, these terms describe the kinetic energy of the electrons, the energy of interactions with fixed ions in the system, image charge potential for an individual electron, changes in potential energy from interaction with $i$ gate electrodes, and a self-energy term that should be renormalized away. Obviously, comparing every experiment to this model is impractical and unnecessary. Section \ref{sec:constant_interaction_model} will introduce the simplest model for capturing quantum dot behavior, called the constant interaction model. In the CI model, the Hamiltonian above is simplified such that conductance through the dot only depends on the energy levels on the dot and the total capacitance of the device, which takes the place of all the electrostatic terms in Equation \ref{eq:full_hamiltonian}. Chapter \ref{sec:CNT} will discuss the basic behavior, at room temperature and low temperatures, of carbon nanotube quantum dot devices in terms of the constant interaction model.

The constant interaction model can be expanded by adding asymmetric, spin dependent tunnel barriers, and spin interactions between electrons in different energy levels confined to the same dot. This is the case when quantum dots are formed using ferromagnetic and superconducting materials. The interplay between magnetism, superconductivity, and transport through a quantum dot is the focus of much of this thesis. 

\section{Working with Carbon Nanotubes}

Carbon nanotube quantum dots are an ideal platform for making the type of transport measurements described above. They are inherently one dimensional conductors with diameters on the order of 1nm. The large aspect ratio of carbon nanotubes (>1000:1) leaves plenty of room to fabricate the metallic contacts needed for transport measurements along the nanotube length.

Building carbon nanotube quantum dot devices tends to be a very personal experience. Each young scientist has a list of recipes, opinions, and techniques both scientific and quasi-religious motivations. Occasionally, this effort results in gorgeous datasets and rich physics. Once this happens, the hundreds of devices built, and subtle techniques tested that lead to the publication are quickly cast aside. The lack of transparency hinders reproducibility and progress in the field. An effort has been made in this thesis to detail all of the techniques used in fabrication and comment on their reproducibility.

\section{Current Work}

Chapters \ref{sec:growth} and \ref{chap:contacts} document a variety of nanotube growth and contact fabrication techniques tested in this work. Chapter \ref{sec:growth} discusses  efforts made in our lab to replicate a number of nanotube growth recipes. In Chapter \ref{chap:contacts}, statistics on the devices made, their fabrication steps, and resulting measurements are discussed. The large volume of devices fabricated allows for a meaningful comparison of fabrication techniques that has not previously been seen. Improved imaging and sample processing techniques are also developed in this chapter. Additionally, a discussion of noise sources in the resulting devices is presented. Recommendations for fabrication are made based on the data analysis.

Chapters \ref{sec:FMCNTQD} and \ref{sec:SCFM} discuss spin-dependent, low-temperature, transport measurements on carbon nanotube quantum dot devices with ferromagnetic and superconducting contacts. These devices allow for the possibility of using applied fields to manipulate quantum states in the device in ways not previously observed. In ferromagnetic devices, we were able to test a number models proposed in previous work and observe, for the first time in F-CNT-F devices, the appearance of conductance suppression and negative differential conductance based on spin selection rules. The F-CNT-S devices show evidence for proximity induced superconductivity as well as spin non-degenerate quantum dot levels in the same device. A measurement of hysteretic conductance switching provides evidence for new transport phenomena in the same devices. These measurements are the first attempt to analyze conductance through such a device.

