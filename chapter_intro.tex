\chapter{Introduction}
\label{sec:intro}
\chaptermark{Introduction}

%This thesis work focuses on a narrow range of fabrication techniques and electronic transport measurements on a specific type of mesoscopic device, carbon nanotube quantum dots. In this introduction, the motivations behind this work will be explained along with an outline of what is to be discussed. 

Using transport spectroscopy to probe low energy density of states can yield a great deal of insight into the nature of the materials being probed. By making a simple conductance measurement, one measures a convolution of the density of states for each of the two materials used in a tunnel junction, $N_1(E)$ and $N_2(E)$.

\begin{equation}
    \label{eq:conductance}
    G \sim \int N_1(E) N_2(E)\frac{df(E+eV)}{dV}dE
\end{equation}

At low temperatures the derivative of the Fermi function is approximated by a delta function and provides a sharp kernel for this convolution. Additionally, at low enough temperatures size confinement becomes significant. Consider an electron confined within a one dimentional length, $L$. The electron has energies on the order of $E \sim \hbar^2k \pi^2/2mL^2$. For length scales smaller than 100nm, $k_B T < E$, and applied voltages less than $E$, the quantum mechanical states in a nanoscale device can be probed through transport spectroscopy.

By combining this powerful technique with ferromagnetic and superconducting materials, transport spectroscopy can be extended to explore a variety of spin dependent phenomenal in nanostructures. 

Carbon nanotube quantum dots are an ideal platform for making the type of transport measurements described above. They are inherently one dimensional conductors with diameters on the order of 1nm. The large aspect ratio of carbon nanotubes (>1000:1) leaves plenty of room to fabricate the metallic contacts needed for transport measurements along the nanotube length.

Building carbon nanotube quantum dot devices tends to be a very personal experience. Each young scientist has a list of recipes, opinions, and techniques with a wide range of scientific and quasi-religious motivations. Occasionally, this effort results in gorgeous datasets and rich physics. Once this happens, the hundreds of devices built, and subtle techniques tested, leading to the publication are quickly forgotten in favor of the results. There is nothing surprising about this, but the lack of transparency hinders reproducibility and progress in the field.

Chapters \ref{sec:growth} and \ref{chap:contacts} document a variety of nanotube growth and contact fabrication techniques tested in this work. Chapter \ref{sec:growth} discusses  efforts made in our lab to replicate a number nanotube growth recipes. In Chapter \ref{chap:contacts}, statistics on the devices made, their fabrication process, and resulting measurements are discussed. The large volume of devices fabricated allows for a meaningful comparison of fabrication techniques that has not previously been seen. Recommendations for fabrication are made on the basis of this data. Additionally, a discussion of noise sources in the resulting devices is presented.

Improvements to the fabrication techniques presented in Chapters \ref{sec:growth} and \ref{chap:contacts} include new imaging techniques, lithography improvements, and recommendations for thin film deposition based on analysis of our devices.

Chapters \label{sec:FMCNTQD} and \label{sec:SCFM} discuss spin-dependent, low-temperature, transport measurements on carbon nanotube quantum dot devices with ferromagnetic and superconducting contacts. In ferromagnetic devices, we were able to test a number models proposed in previous work and observe the appearance of conductance fluctuations based on spin selection rules. The F-CNT-S show evidence of proximity induced superconductivity as well as magnetic field dependent fluctuations in transport through the quantum dot. These measurements are the first attempt to analyze conductance through such a device.

